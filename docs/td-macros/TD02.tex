% partie d�clarative
\documentclass[a4paper,10pt]{article}
%\pdfoptionpdfminorversion = 5

\usepackage[english]{babel} %,francais
%\usepackage[latin1]{inputenc}
%\usepackage[T1]{fontenc}\usepackage{amsfonts}
\usepackage{amssymb}
\usepackage{amsmath}
\usepackage{amsthm}
\usepackage{bbm}
\usepackage{stmaryrd}
\usepackage{enumerate}
%usepackage{url}
%usepackage{wasysym}
\usepackage[colorlinks=true,urlcolor=blue,pdfstartview=FitH]{hyperref} 
%usepackage{lscape}
%usepackage{manfnt}
%usepackage{mathbbol}
%usepackage{eurosym}
\usepackage{color}
%\usepackage{bbding}
\usepackage[cyr]{aeguill}     % Police vectorielle TrueType, guillemets fran�ais
\usepackage[pdftex]{graphicx} % Pour l'insertion d'images


%\usepackage{aertt}
\usepackage{harvard}
\usepackage{cje}

%\usepackage[cyr]{aeguill}     % Police vectorielle TrueType, guillemets fran�ais
\usepackage[pdftex]{graphicx} % Pour l'insertion d'images
\DeclareGraphicsExtensions{.jpg,.mps,.pdf,.png,.bmp}%TCIDATA{OutputFilter=LATEX.DLL}
%TCIDATA{Version=4.10.0.2345}
%TCIDATA{Created=Thursday, August 25, 2005 14:35:37}
%TCIDATA{LastRevised=Tuesday, October 18, 2005 15:55:08}
%TCIDATA{<META NAME="GraphicsSave" CONTENT="32">}
%TCIDATA{<META NAME="DocumentShell" CONTENT="Standard LaTeX\Blank - Standard LaTeX Article">}
%TCIDATA{CSTFile=40 LaTeX article.cst}

%\newtheorem{theorem}{Theorem}
%\newtheorem{acknowledgement}[theorem]{Acknowledgement}
%\newtheorem{algorithm}[theorem]{Algorithm}
%\newtheorem{axiom}[theorem]{Axiom}
%\newtheorem{case}[theorem]{Case}
%\newtheorem{claim}[theorem]{Claim}
%\newtheorem{conclusion}[theorem]{Conclusion}
%\newtheorem{condition}[theorem]{Condition}
%\newtheorem{conjecture}[theorem]{Conjecture}
%\newtheorem{corollary}[theorem]{Corollary}
%\newtheorem{criterion}[theorem]{Criterion}
%\newtheorem{definition}[theorem]{Definition}
%\newtheorem{example}[theorem]{Example}
%\newtheorem{exercise}[theorem]{Exercise}
%\newtheorem{lemma}[theorem]{Lemma}
%\newtheorem{notation}[theorem]{Notation}
%\newtheorem{problem}[theorem]{Problem}
%\newtheorem{proposition}[theorem]{Proposition}
%\newtheorem{remark}[theorem]{Remark}
%\newtheorem{solution}[theorem]{Solution}
%\newtheorem{summary}[theorem]{Summary}
%\newenvironment{proof}[1][Proof]{\noindent\textbf{#1.} }{\ \rule{0.5em}{0.5em}}
%\input{tcilatex}

\newcommand{\EnTete}[3]{
		\begin{flushleft}
			Fran\c{c}ois Le Grand \hfill #1
			
			legrand@pse.ens.fr
			
			Cours de Macro�conomie 4 (Prof. Daniel Cohen)
		\end{flushleft}
		\begin{center}	
			\bigskip
			TD {#2}
			
			\bigskip
			\textbf{#3} 
		\end{center}

\hrule\vspace{\baselineskip} 
}
%\noindent
\renewcommand{\thesection}{\Alph{section}}
\setlength{\parindent}{0pt}

\makeatletter
\newcounter{question}
\newcounter{subquestion}
\newcommand{\thenumq}{\arabic{question}}
\newcommand{\thenumsq}{\alph{subquestion}}
\newcommand{\ques}{\stepcounter{question}{\hskip.4cm{\thenumq.~\,}}}
\newcommand{\subques}{\stepcounter{subquestion}{\hskip.8cm{\thenumsq.~}}}
\@addtoreset{subquestion}{question}
\makeatother



\newcommand{\textenitalique}[1]{
\bigskip
\textit{#1} \\\nopagebreak
}


\begin{document}

\EnTete{S�ances des 3--10 et 6--13 Octobre 2006}{2}{Architecture Optimale de la Protection de l'Emploi}

\bibliographystyle{cje}
\bibliography{Biblio_TD}
\nocite{BlTi:04}

\paragraph{\large{Points techniques du TD :}} 
\begin{itemize}
	\item Al�a moral,
	\item Coh�rence temporelle. 
\end{itemize}

\section{Set-up du mod\`{e}le}

\textenitalique{Hypoth\`{e}ses}

\begin{itemize}
\item L'\'{e}conomie est constitu\'{e}e d'un continuum de travailleurs de
masse $1$ et d'un continuum d'entrepreneurs de masse $1$.

\item Les entrepreneurs sont neutres au risque. Chaque entrepreneur peut
mener un projet (cr\'{e}er une entreprise). Il y a un co\^{u}t fixe au
lancement d'un projet $I$, identique pour tous les entrepreneurs. Si un
projet est lanc\'{e}, un travailleur est employ\'{e} et la productivit\'{e}
du \textquotedblleft match\textquotedblright\ entrepreneur-travailleur $(y)$
est r\'{e}v\'{e}l\'{e}e. La productivit\'{e} $(y)$ d'un match est tir\'{e}e
d'une fonction de distribution dont la fonction de r\'{e}partition est $G(y)$
continue et diff\'{e}rentiable (on note $g(y)$ dans $[0,1]$ la densit\'{e}
associ\'{e}e). Une fois la productivit\'{e} r\'{e}v\'{e}l\'{e}e,
l'entreprise peut soit produire et r\'{e}mun\'{e}rer le travailleur, soit le
licencier (lequel se retrouve au ch\^{o}mage).

\item Les travailleurs sont averses au risque: leur fonction d'utilit\'{e}
est $U(x)$ concave, o\`{u} $x$ est le revenu du travailleur. En l'absence
d'allocation ch\^{o}mage, le revenu de r\'{e}serve d'un travailleur sans
emploi est $b$.

\item Le dernier agent de cet \'{e}conomie est l'Etat: l'Etat finance les
allocations-ch\^{o}mage $(\mu )$ \`{a} budget \'{e}quilibr\'{e}.\ Pour
financer la caisse d'allocations, il peut avoir recours \`{a} deux
instruments: taxe au licenciement d'un travailleur $\left( f\right) $ pay%
\'{e}es \`{a} chaque licenciement ou cotisations sociales $\left( \tau
\right) $ pay\'{e}es par la firme pour chaque travailleur employ\'{e}.
\end{itemize}

\textenitalique{Timing}

\begin{itemize}
\item P\'{e}riode 0: L'Etat choisit le triplet $\left\{ f,\tau ,\mu \right\} 
$ sous-contrainte de budget \'{e}quilibr\'{e}.

\item P\'{e}riode 1: Les entrepreneurs d\'{e}cident de lancer un projet,
paient le co\^{u}t fixe $(I)$ et emploient des travailleurs.\ Les
entrepreneurs offrent un contrat $\left( w,y^{\ast }\right) $ qui stipule un
niveau de salaire fix\'{e} \textit{ex-ante} $w$ et un niveau minimum de
productivit\'{e} $(y^{\ast })$ en de\c{c}a duquel le travailleur est licenci%
\'{e}. Noter que les firmes ne proposent pas de salaires contingents \`{a}
la r\'{e}alisation $(w(y))$ du fait de l'aversion au risque des
travailleurs. Noter aussi que comme les entrepreneurs font face au m\^{e}me
co\^{u}t fixe et \`{a} la m\^{e}me distribution de productivit\'{e}, \`{a} l'%
\'{e}quilibre, ils chercheront tous \`{a} lancer un projet.

\item P\'{e}riode 2: L'incertitude sur la productivit\'{e} est r\'{e}v\'{e}l%
\'{e}e: si $y>y^{\ast }$, la production est r\'{e}alis\'{e}e et le salaire $%
\left( w\right) $ est pay\'{e} au travailleur et la cotisation sociale $%
\left( \tau \right) $ est pay\'{e} \`{a} l'Etat; si $y<y^{\ast }$, la firme
paie la taxe $\left( f\right) $ et le travailleur est licenci\'{e} et touche 
$\left( \mu \right) $ de l'Etat.
\end{itemize}

\medskip

L'objet du TD est de trouver l'architecture optimale de la protection de
l'emploi, \textit{i.e }le triplet $\left\{ f,\tau ,\mu \right\} $ optimal
dans divers cas de figure.

\section{Benchmark}

\ques Ecrire l'utilit\'{e} esp\'{e}r\'{e} d'un travailleur $(V_{W})$
conditionnellement au contrat $\left( w,y^{\ast }\right) $.

\medskip

\ques Ecrire le gain esp\'{e}r\'{e} d'un entrepreneur $(V_{F})$
conditionnellement au contrat $\left( w,y^{\ast }\right) $. A quelle
condition, les entrepreneurs lancent un projet~?

\medskip

\ques Ecrire la contrainte budg\'{e}taire du gouvernement.

\medskip

\textenitalique{Programme de la firme}

\ques Ecrire le programme de maximisation de la firme (N'oubliez pas la
contrainte de participation des travailleurs~!). Donner les conditions du
premier-ordre. En d\'{e}duire que:%
\begin{equation*}
y^{\ast }=w+\tau -f-\left( \frac{U(w)-U(b+\mu )}{U^{\prime }(w)}\right)
\end{equation*}

Commenter.

\medskip

\ques Pourquoi cette solution n'est pas coh\'{e}rente temporellement~? Donner la
solution du programme pr\'{e}c\'{e}dent coh\'{e}rente temporellement.

\medskip

On supposera par la suite que le contrat $\left( w,y^{\ast }\right) $ est coh%
\'{e}rent temporellement.

\medskip

\textenitalique{Optimum social}

Un planificateur bienveillant cherche \`{a} maximiser l'utilit\'{e} esp\'{e}r%
\'{e}e des travailleurs sous les contraintes suivantes:
\begin{itemize}
	\item contrainte budg\'{e}taire du gouvernement,
	\item contrainte de participation des entrepreneurs,
	\item seuil de productivit\'{e} choisi par les entreprises (et coh\'{e}rent
temporellement).
\end{itemize}

\medskip

\ques Montrer que les contraintes de participation des entrepreneurs et d'\'{e}%
quilibre budg\'{e}taire implique la contrainte suivante:
\begin{equation*}
-G(y^{\ast })\mu +\int_{y^{\ast }}^{\infty }ydG(y)-\left( 1-G(y^{\ast
})\right) w\geq I
\end{equation*}

On suppose que le planificateur choisit d'abord $\{w,\mu ,y^{\ast }\}$ de
sorte \`{a} maximiser:
\begin{eqnarray*}
&&\max_{\{w,\mu ,y^{\ast }\}}\left\{ G(y^{\ast })U(b+\mu )+\left(
1-G(y^{\ast })\right) U(w)\right\} \\
s.c &:&\text{ }-G(y^{\ast })\mu +\int_{y^{\ast }}^{\infty }ydG(y)-\left(
1-G(y^{\ast })\right) w\geq I
\end{eqnarray*}

(En effet, le planificateur pourra toujours fixer le couple $\{f,\tau \}$
tel que $y^{\ast }$ corresponde au seuil choisi par les entreprises et tel
que la contrainte budg\'{e}taire de l'Etat est \'{e}quilibr\'{e}e).

En d\'{e}duire le niveau d'allocations-ch\^{o}mage vers\'{e}es par le
gouvernement et le seuil de productivit\'{e} \`{a} l'optimum social en
fonction de $w$ et $b$. Commenter.

\medskip

\ques En d\'{e}duire le design $\{f,\tau \}$ choisi par le gouvernement pour r%
\'{e}aliser l'optimum social. Commenter.

\section{Limites de l'assurance compl\`{e}te: al\'{e}a moral}

Dans le mod\`{e}le benchmark, les travailleurs sont parfaitement assur\'{e}%
s: en pratique, ce cas est peu r\'{e}aliste car une parfaite assurance d\'{e}%
sincite les travailleurs \`{a} fournir l'effort maximal pour chercher un
emploi et/ou garder leur emploi (al\'{e}a moral). Nous allons voir dans
cette deuxi\`{e}me partie comment ces questions d'incitations modifient le
design de la protection de l'emploi.

\medskip

Nous supposons que la contrainte d'incitation des travailleurs (\`{a}
fournir un effort lorsqu'ils sont employ\'{e}s) peut se r\'{e}\'{e}crire de
la mani\`{e}re suivante:%
\begin{equation*}
\left( 1-G(y^{\ast })\right) \left( U(w)-U(b+\mu )\right) \geq B\text{ \ \ \
\ \ \ \ \ \ \ \ \ \ \ \ \ \ \ \ }(IW)
\end{equation*}

o\`{u} $B$ d\'{e}signe les b\'{e}n\'{e}fices priv\'{e}s d'un travailleur
lorsque celui -ci ne fournit pas d'effort.

\medskip

\ques Interpr\'{e}ter la condition $(IW)$

\medskip

\textenitalique{Optimum social}

\ques R\'{e}\'{e}crire le programme du planificateur social. Montrer que

\subques l'assurance n'est plus compl\`{e}te

\subques la valeur seuil v\'{e}rifie $y^{\ast }=b+\left[ w-(b+\mu )-\frac{%
U(w)-U(b+\mu )}{U^{\prime }(w)}\right] $

Commenter.

\medskip

\ques En d\'{e}duire le design $\{f,\tau \}$ choisi par le gouvernement pour r%
\'{e}aliser l'optimum social en fonction de $w$ et des param\`{e}tres du mod%
\`{e}le. Commenter.

\section{Shallow Pockets}

Jusqu'alors, nous avons suppos\'{e} que les firmes peuvent toujours financer
les taxes de licenciements (pas de contraintes financi\`{e}res,
\textquotedblleft Deep Pockets\textquotedblright ). Nous \'{e}tudions
maintenant le design optimal de la protection de l'emploi lorsque les firmes
sont contraintes financi\`{e}rement.

Nous supposons que la richesse d'un entrepreneur $(W_{E})$ avant de cr\'{e}%
er sa firme est born\'{e}e:

\begin{equation*}
W_{E}\leq I+f^{\ast }
\end{equation*}

\medskip

\ques Quelle est la limite impos\'{e}e \`{a} $f$ dans ce cas~? On supposera que
cette contrainte est satur\'{e}e\footnote{%
Dans le cas contraire, le probl\`{e}me est \'{e}quivalent \`{a} la partie A.}%
. En d\'{e}duire que le choix de $\mu $ de l'Etat est contraint par:

\begin{equation*}
G(y^{\ast })\mu \leq f^{\ast }+\left( 1-G(y^{\ast })\right) (y^{\ast }-w)
\end{equation*}

\medskip

\textenitalique{Optimum social}

\ques Ecrire le programme de maximisation du planificateur social en tenant
compte de cette nouvelle contrainte sur le niveau d'allocations-ch\^{o}mage
et d\'{e}river les conditions du premier ordre (on ne demande pas de r\'{e}%
soudre).\ Montrer que:

\subques \`{a} l'optimum, il y a assurance compl\`{e}te

\subques le seuil de productivit\'{e} en de\c{c}a duquel il y a licenciement est
plus \'{e}lev\'{e} qu'en~A.

\medskip

\ques En d\'{e}duire le design $\{f,\tau \}$ choisi par le gouvernement pour r%
\'{e}aliser l'optimum social en fonction de $y^{\ast }$ et des param\`{e}%
tres du mod\`{e}le. Commenter.

\end{document}

