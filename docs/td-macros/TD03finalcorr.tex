% partie d�clarative
\documentclass[a4paper,10pt]{article}
%\pdfoptionpdfminorversion = 5

\usepackage[english]{babel} %,francais
%\usepackage[latin1]{inputenc}
%\usepackage[T1]{fontenc}\usepackage{amsfonts}
\usepackage{amssymb}
\usepackage{amsmath}
\usepackage{amsthm}
\usepackage{bbm}
\usepackage{stmaryrd}
\usepackage{enumerate}
%usepackage{url}
%usepackage{wasysym}
\usepackage[colorlinks=true,urlcolor=blue,pdfstartview=FitH]{hyperref} 
%usepackage{lscape}
%usepackage{manfnt}
%usepackage{mathbbol}
%usepackage{eurosym}
\usepackage{color}
%\usepackage{bbding}
\usepackage[cyr]{aeguill}     % Police vectorielle TrueType, guillemets fran�ais
\usepackage[pdftex]{graphicx} % Pour l'insertion d'images


%\usepackage{aertt}
\usepackage{harvard}
\usepackage{cje}

%\usepackage[cyr]{aeguill}     % Police vectorielle TrueType, guillemets fran�ais
\usepackage[pdftex]{graphicx} % Pour l'insertion d'images
\DeclareGraphicsExtensions{.jpg,.mps,.pdf,.png,.bmp}%TCIDATA{OutputFilter=LATEX.DLL}
%TCIDATA{Version=4.10.0.2345}
%TCIDATA{Created=Thursday, August 25, 2005 14:35:37}
%TCIDATA{LastRevised=Tuesday, October 18, 2005 15:55:08}
%TCIDATA{<META NAME="GraphicsSave" CONTENT="32">}
%TCIDATA{<META NAME="DocumentShell" CONTENT="Standard LaTeX\Blank - Standard LaTeX Article">}
%TCIDATA{CSTFile=40 LaTeX article.cst}

%\newtheorem{theorem}{Theorem}
%\newtheorem{acknowledgement}[theorem]{Acknowledgement}
%\newtheorem{algorithm}[theorem]{Algorithm}
%\newtheorem{axiom}[theorem]{Axiom}
%\newtheorem{case}[theorem]{Case}
%\newtheorem{claim}[theorem]{Claim}
%\newtheorem{conclusion}[theorem]{Conclusion}
%\newtheorem{condition}[theorem]{Condition}
%\newtheorem{conjecture}[theorem]{Conjecture}
%\newtheorem{corollary}[theorem]{Corollary}
%\newtheorem{criterion}[theorem]{Criterion}
%\newtheorem{definition}[theorem]{Definition}
%\newtheorem{example}[theorem]{Example}
%\newtheorem{exercise}[theorem]{Exercise}
%\newtheorem{lemma}[theorem]{Lemma}
%\newtheorem{notation}[theorem]{Notation}
%\newtheorem{problem}[theorem]{Problem}
%\newtheorem{proposition}[theorem]{Proposition}
%\newtheorem{remark}[theorem]{Remark}
%\newtheorem{solution}[theorem]{Solution}
%\newtheorem{summary}[theorem]{Summary}
%\newenvironment{proof}[1][Proof]{\noindent\textbf{#1.} }{\ \rule{0.5em}{0.5em}}
%\input{tcilatex}

\newcommand{\EnTete}[3]{
		\begin{flushleft}
			Fran\c{c}ois Le Grand \hfill #1
			
			legrand@pse.ens.fr
			
			Cours de Macro�conomie 4 (Prof. Daniel Cohen)
			\url{http://www.pse.ens.fr/junior/legrand/cours.html}
		\end{flushleft}
		\begin{center}	
			\medskip
			\textbf{TD {#2}}
			
			\bigskip
			\textbf{#3} 
		\end{center}

\hrule\vspace{\baselineskip} 
}
%\noindent
\renewcommand{\thesection}{\Alph{section}}
\setlength{\parindent}{0pt}

\makeatletter
\newcounter{question}
\newcounter{subquestion}
\newcommand{\thenumq}{\arabic{question}}
\newcommand{\thenumsq}{\alph{subquestion}}
\newcommand{\ques}{\stepcounter{question}{\hskip.4cm{\thenumq.~\,}}}
\newcommand{\subques}{\stepcounter{subquestion}{\hskip.8cm{\thenumsq.~}}}
\@addtoreset{subquestion}{question}
\makeatother



\newcommand{\textenitalique}[1]{
\bigskip
\textit{#1} \\\nopagebreak
}

\newenvironment{changemargin}[2]{\begin{list}{}{%
\setlength{\topsep}{0pt}%
\setlength{\leftmargin}{0pt}%
\setlength{\rightmargin}{0pt}%
\setlength{\listparindent}{\parindent}%
\setlength{\itemindent}{\parindent}%
\setlength{\parsep}{0pt plus 1pt}%
\addtolength{\leftmargin}{#1}%
\addtolength{\rightmargin}{#2}%
}\item }{\end{list}}

\newcommand{\correction}[1]{
\begin{changemargin}{1cm}{0cm}
   \it #1
\end{changemargin} }

\begin{document}

\EnTete{S�ances des 6--13 Octobre 2006}{3}{Equivalence ricardienne et d\'{e}mographie}

\bibliographystyle{cje}
\bibliography{Biblio_TD}
\nocite{Bu:88}

Il y a \'{e}quivalence ricardienne dans une \'{e}conomie si la mani\`{e}re
dont l'Etat finance ses d\'{e}penses n'a pas d'effet r�el. Il est �quivalent que l'\'Etat se finance par imp�t ou par dette. L'argument est le suivant. Si l'\'Etat se finance par �mission de dette, les agents anticipent parfaitement l'augmentation de taxes futures et ce faisant diminuent leur consommation de la m�me fa�on.   \bigskip
Les hypoth�ses sous--jacentes � la validit� de l'�quivalence ricardienne sont les suivantes~:
\begin{enumerate}[(i)~]
	\item Horizon de vie infini : sinon ce sont d'autres agents qui rembourseront la dette\ldots
	\item Taxes {\it lump--sum} : sinon les taxes cr�ent des distorsions sur le comportement des agents et cela implique que le timing des taxes comptent.
	\item Pas d'environnment stochastique : sinon l'anticipation des agents diff�re de la r�alisation.
\end{enumerate}

Ainsi, le mod\`{e}le avec agent repr\'{e}sentatif \`{a} horizon de vie
infini et le mod\`{e}le \`{a} g\'{e}n\'{e}rations imbriqu\'{e}es avec agents
vivant deux p\'{e}riodes ont des implications tr\`{e}s diff\'{e}rentes : l'%
\'{e}quivalence ricardienne tient dans le premier cadre mais le "timing" des
imp\^{o}ts n'est pas indiff\'{e}rent dans le second.

Une mani\`{e}re de retrouver l'\'{e}quivalence dans un mod\`{e}le \`{a} g%
\'{e}n\'{e}rations est de consid\'{e}rer altruisme, leg et comportement
dynastique. Les agents qui ont contract� la dette doivent se sentir `responsables' de son remboursement. 
\bigskip

Dans cemod�le, les individus ne sont pas altruistes et ont une probabilit� de cd�c�s exog�ne, ind�pendante de leur �ge. 

\bigskip

\section{D\'{e}mographie}

\ques Calculer $n$ le taux de croissance de la population.

\correction{
\begin{equation*}
N_{t+1}=(1+n)N_{t}=(1-p)N_{t}+bN_{t+1}
\end{equation*}%
\begin{equation*}
\Rightarrow 1+n=1-p+b(1+n)
\end{equation*}%
\begin{equation*}
\Rightarrow 1+n=\frac{1-p}{1-b}
\end{equation*}%
\medskip
}

2. Donner la probabilit\'{e} pour un agent n\'{e} \`{a} la p\'{e}riode $s$
meure avant le d\'{e}but de la p\'{e}riode $t$ $>s$.

\correction{La probabilit\'{e} pour un agent n\'{e} en p\'{e}riode $s$ d'\^{e}tre en
vie \`{a} la p\'{e}riode $s+1$ est $(1-p)$ et par r\'{e}currence, sa
probabilit\'{e} d'\^{e}tre en vie \`{a} la p\'{e}riode $t>s$ est est $%
(1-p)^{t-s}$. On suppose que la loi des grands nombres s'applique de sorte
que $(1-p)^{t-s}$ est aussi la part de la cohorte de la p\'{e}riode $s$
encore en vie au d\'{e}but de la p\'{e}riode $t$.}

\section{Dotations, syst\`{e}me viager et \'{e}volution de la richesse
financi\`{e}re }

On note $a_{s,t}$\ la richesse financi\`{e}re \`{a} la p\'{e}riode $t$ d'un
agent n\'{e} \`{a} la p\'{e}riode $s$. Un agent commence avec une richesse
financi\`{e}re $a_{s,s}=0$. A chaque p\'{e}riode, il re\c{c}oit un salaire $%
w_{t}$ et paie un montant d'imp\^{o}ts $\tau _{t}$. A la p\'{e}riode $t$, un
agent dispose d'un montant $a_{s,t}+w_{t}-\tau _{t}$\ qu'il peut soit
consommer, soit placer.

Le placement de la p\'{e}riode $t$ est doublement r\'{e}mun\'{e}r\'{e} \`{a}
la p\'{e}riode $t+1$ :

\begin{itemize}
	\item au taux d'int\'{e}r\^{e}t r\'{e}el $r$, constant au cours du temps,
	\item via une prime proportionnelle \`{a} son \'{e}pargne.
\end{itemize}

\medskip

On suppose en effet qu'il existe un syst\`{e}me viager parfaitement
concurrentiel (i.e. profit nul) dont le principe est le suivant : \`{a}
chaque p\'{e}riode un agent re\c{c}oit une prime proportionnelle \`{a} son 
\'{e}pargne; en \'{e}change, il r\'{e}troc\`{e}de l'int\'{e}gralit\'{e} de
sa richesse financi\`{e}re au jour de sa mort.\bigskip

\ques Ecrire la loi d'\'{e}volution de la richesse financi\`{e}re d'un individu
n\'{e} \`{a} la p\'{e}riode $s$ pendant la dur\'{e}e de sa vie, \`{a} taux
assurantiel $\Pi $\ donn\'{e}.\medskip

\begin{eqnarray*}
a_{s,t+1} &=&(1+r)(a_{s,t}+w_{t}-\tau _{t}-c_{t})+\Pi (a_{s,t}+w_{t}-\tau
_{t}-c_{t}) \\
&=&(1+r+\Pi )(a_{s,t}+w_{t}-\tau _{t}-c_{t})
\end{eqnarray*}

2. \ques Ecrire l'ensemble des d\'{e}penses et des recettes pour le syst\`{e}me
viager et d\'{e}terminer le taux $\Pi $ v\'{e}rifiant la condition de profit
nul.\medskip
\correction{
On suppose que le syst\`{e}me viager est concurrentiel (hypoth�se libre entr�e).

-- d\'{e}penses en p\'{e}riode $t$ : 
\begin{equation*}
\sum_{s<t}(1-p)^{t-s}\Pi (a_{s,t-1}+w_{t-1}-\tau _{t-1}-c_{t-1})bN_{s}
\end{equation*}

-- recettes en p\'{e}riode $t$ : 
\begin{equation*}
\sum_{s<t}p(1-p)^{t-1-s}(1+r)(a_{s,t-1}+w_{t-1}-\tau _{t-1}-c_{t-1})bN_{s}
\end{equation*}

Condition de profit nul implique : 
\begin{equation*}
(1-p)\Pi =p(1+r)
\end{equation*}
\begin{equation*}
\Rightarrow \Pi =(1+r)\frac{p}{1-p}
\end{equation*}
}
3. R\'{e}mun\'{e}ration totale de l'\'{e}pargne%
\begin{eqnarray*}
1+r_{h} &=&1+r+\Pi =1+r+(1+r)\frac{p}{1-p} \\
&=&\frac{1+r}{1-p}
\end{eqnarray*}

\bigskip

\textbf{III -- Choix optimal de consommation\medskip }

Cette partie introduit \`{a} la th\'{e}orie du revenu permanent dans le cas o%
\`{u} la dur\'{e}e de vie est incertaine.

\bigskip

1. Programme :%
\begin{equation*}
\max \quad \sum_{t=s}^{\infty }[{(1-p)\beta ]^{t-s}\log (c_{s,t})}
\end{equation*}%
\begin{equation*}
s.c.\quad a_{s,t+1}=(1+r_{h})(a_{s,t}+w_{t}-\tau _{t}-c_{t})
\end{equation*}%
\begin{equation*}
\lim_{t\rightarrow \infty }\quad \frac{a_{s,t}}{(1+r_{h})^{t}}=0
\end{equation*}%
\begin{equation*}
a_{s,s}=0
\end{equation*}

On normalise les multiplicateurs de Lagrange de sorte que 
\begin{equation*}
\pounds =\sum_{t=s}^{\infty }[{(1-p)\beta ]^{t-s}\{\log (c_{s,t})+\lambda }%
_{s,t}[(1+r_{h})(a_{s,t}+w_{t}-\tau _{t}-c_{s,t})-a_{s,t+1}]\}
\end{equation*}%
\medskip

2. Conditions du premier ordre par rapport \`{a} $c_{s,t}$\ et \`{a} $%
a_{s,t+1}$\ : 
\begin{equation*}
u^{\prime }(c_{s,t})=\lambda _{s,t}.(1+r_{h})
\end{equation*}

\begin{equation*}
\lambda _{s,t+1}(1+r_{h})\beta (1-p)=\lambda _{s,t}
\end{equation*}

D'o\`{u} 
\begin{equation*}
\frac{c_{s,t+1}}{c_{s,t}}=\frac{u^{\prime }(c_{s,t})}{u^{\prime }(c_{s,t+1})}%
=\frac{\lambda _{s,t}}{\lambda _{s,t+1}}=(1+r_{h})\beta (1-p)=\beta (1+r)
\end{equation*}%
\medskip

3. Contrainte budg\'{e}taire intertemporelle 
\begin{equation*}
a_{s,t+1}=(1+r_{h})(a_{s,t}+w_{t}-\tau _{t}-c_{s,t})
\end{equation*}%
\begin{equation*}
\underset{s.c.\;Ponzi}{\Rightarrow }\sum_{i=0}^{\infty }\frac{c_{s,t+i}}{%
(1+r_{h})^{i}}=a_{s,t}+\sum_{i=0}^{\infty }\frac{(w_{t+i}-\tau _{t+i})}{%
(1+r_{h})^{i}}=a_{s,t}+h_{t}
\end{equation*}%
\medskip

4. Fonction de consommation \bigskip

L'\'{e}quation d'Euler implique :

\begin{equation*}
c_{s,t+i}=[\beta (1+r)]^{i}c_{s,t}
\end{equation*}

(NB : dans le cas g\'{e}n\'{e}ral, lorsqu'on suppose que la condition du
premier ordre est v\'{e}rifi\'{e}e, cela veut dire qu'on fait implicitement
l'hypoth\`{e}se que les march\'{e}s financiers sont parfaits et qu'il n'y a
pas de contraintes de liquidit\'{e}).\medskip

Donc la contrainte budg\'{e}taire s'\'{e}crit :%
\begin{equation*}
c_{s,t}\sum_{i=0}^{\infty }\left( \frac{\beta (1+r)}{1+r_{h}}\right)
^{i}=a_{s,t}+h_{t}
\end{equation*}%
\medskip

La consommation de la p\'{e}riode $t$ d'un agent n\'{e} \`{a} la p\'{e}riode 
$s$ est donc :

\begin{equation*}
c_{s,t}=\Omega (a_{s,t}+h_{t})
\end{equation*}

avec 
\begin{equation*}
\Omega =1-\beta (1-p)
\end{equation*}

Dans le cas standard d'un agent avec horizon infini sans mortalit\'{e}, on a
une propension marginale \`{a} consommer la richesse totale (i.e. humaine +
financi\`{e}re) \'{e}gale \`{a} $1-\beta $. Cette fonction de consommation
n'a rien \`{a} voir avec la fonction de consommation "keyn\'{e}sienne" selon
laquelle les agents consomment \`{a} proportion de leur revenu disponible
courant. Ici consommation et revenu courant sont possiblement compl\'{e}%
tement d\'{e}connect\'{e}s.\medskip

\bigskip

5. Consommation agr\'{e}g\'{e}e

Pour calculer la consommation agr\'{e}g\'{e}e de la date $t$, on doit agr%
\'{e}ger sur les diff\'{e}rentes cohortes pass\'{e}es, en prenant en compte
le fait qu'elles ont diminu\'{e} de taille depuis leur apparition.

Il est imm\'{e}diat que la taille en $t$ de la cohorte n\'{e}e \`{a} la p%
\'{e}riode $s$ est $(1-p)^{t-s}bN_{s}$. Donc :

\begin{eqnarray*}
C_{t} &=&\sum_{s\leq t}c_{s,t}(1-p)^{t-s}bN_{s} \\
&=&\Omega \sum_{s\leq t}(a_{s,t}+h_{t})(1-p)^{t-s}bN_{s} \\
&=&\Omega \sum_{s\leq t}a_{s,t}(1-p)^{t-s}bN_{s}+\Omega \sum_{s\leq
t}h_{t}(1-p)^{t-s}bN_{s}
\end{eqnarray*}

\bigskip \bigskip

\textbf{IV -- Finances publiques}

\bigskip

1. Evolution de la valeur de la dette 
\begin{equation*}
\frac{D_{t+1}}{1+r}=D_{t}+G_{t}-T_{t}
\end{equation*}%
\medskip

2. Equilibre des finances publiques \`{a} long terme\medskip

La loi d'\'{e}volution de la dette entre $t$ et $T$ implique :%
\begin{equation*}
D_{T}=(1+r)^{T}D_{t}+\sum_{i=0}^{T-t}(1+r)^{T-i}(G_{t+i}-T_{t+i})
\end{equation*}%
\begin{equation*}
\frac{D_{T}}{(1+r)^{T}}=D_{t}+\sum_{i=0}^{T-t}\frac{G_{t+i}-T_{t+i}}{%
(1+r)^{i}}
\end{equation*}%
\bigskip

\begin{eqnarray*}
\lim_{T\rightarrow \infty }\frac{D_{T}}{(1+r)^{T}} &=&0\quad \Rightarrow
\sum_{i=0}^{\infty }\frac{T_{t+i}}{(1+r)^{i}}=D_{t}+\sum_{i=0}^{\infty }%
\frac{G_{t+i}}{(1+r)^{i}} \\
&\Rightarrow &D_{t}=\sum_{i=0}^{\infty }\frac{(T_{t+i}-G_{t+i})}{(1+r)^{i}}
\end{eqnarray*}

\bigskip

\textbf{V -- Equilibre et condition pour la neutralit\'{e} du chemin fiscal}

\bigskip

1. La contre-partie de la richesse financi\`{e}re agr\'{e}g\'{e}e sur tous
les agents priv\'{e}s, c'est n\'{e}cessairement la dette de l'Etat.%
\begin{equation*}
A_{t}=D_{t}
\end{equation*}%
\begin{eqnarray*}
C_{t} &=&\Omega (A_{t}+H_{t}) \\
&=&\Omega (D_{t}+H_{t})
\end{eqnarray*}%
\medskip

2. Prise en compte de la contrainte budg\'{e}taire intertemporelle de l'Etat
dans la consommation agr\'{e}g\'{e}e\medskip

On a :

\begin{equation*}
D_{t}=\sum_{i=0}^{\infty }\frac{(T_{t+i}-G_{t+i})}{(1+r)^{i}}
\end{equation*}

Donc : 
\begin{equation*}
C_{t}=\Omega \left( \sum_{i=0}^{\infty }\frac{(T_{t+i}-G_{t+i})}{(1+r)^{i}}%
+H_{t}\right)
\end{equation*}%
\medskip

3. On fait appara\^{\i}tre dans la richesse humaine les imp\^{o}ts futurs
:\smallskip

\begin{eqnarray*}
H_{t} &=&h_{t}N_{t} \\
&=&\sum_{i=0}^{\infty }\frac{w_{t+i}}{(1+r_{h})^{i}}N_{t}-\sum_{i=0}^{\infty
}\frac{\tau _{t+i}}{(1+r_{h})^{i}}N_{t} \\
&=&\sum_{i=0}^{\infty }\frac{w_{t+i}}{(1+r_{h})^{i}}N_{t}-\sum_{i=0}^{\infty
}\frac{\tau _{t+i}N_{t+i}}{(1+r_{h})^{i}}\frac{N_{t}}{N_{t+i}} \\
&=&\sum_{i=0}^{\infty }\frac{w_{t+i}}{(1+r_{h})^{i}}N_{t}-\sum_{i=0}^{\infty
}\frac{T_{t+i}}{[(1+r_{h})(1+n)]^{i}}
\end{eqnarray*}

Dans la consommation agr\'{e}g\'{e}e, les imp\^{o}ts futurs agr\'{e}g\'{e}s
sont escompt\'{e}s au taux :

\begin{equation*}
(1+r_{h})(1+n)=\frac{1+r}{1-p}\frac{1-p}{1-b}=\frac{1+r}{1-b}
\end{equation*}

Nota bene : pour $b>0$, ce taux d'escompte est inf\'{e}rieur \`{a} $r$.

\newpage

La consommation agr\'{e}g\'{e}e peut donc s'\'{e}crire :\smallskip

\begin{equation*}
C_{t}=\Omega \left\{ \sum_{i=0}^{\infty }\frac{(T_{t+i}-G_{t+i})}{(1+r)^{i}}%
+\sum_{i=0}^{\infty }\frac{w_{t+i}N_{t}}{(1+r_{h})^{i}}-\sum_{i=0}^{\infty }%
\frac{T_{t+i}}{[(1+r_{h})(1+n)]^{i}}\right\} 
\end{equation*}%
soit :%
\begin{equation*}
C_{t}=\Omega \left\{ -\sum_{i=0}^{\infty }\frac{G_{t+i}}{(1+r)^{i}}%
+\sum_{i=0}^{\infty }\frac{w_{t+i}N_{t}}{(1+r_{h})^{i}}+\sum_{i=0}^{\infty }%
\left[ \frac{1}{(1+r)^{i}}-\frac{1}{(\frac{1+r}{1-b})^{i}}\right]
T_{t+i}\right\} 
\end{equation*}

\bigskip

Si on compare deux sentiers de taxation $\{T_{t+i}^{1}\}_{i\geq t}$\ et $%
\{T_{t+i}^{2}\}_{i\geq t}$\ qui permettent de financer un m\^{e}me sentier
de d\'{e}penses $\{G_{t+i}\}_{i\geq t}$, les sentiers de consommation
correspondants sont identiques si et seulement si $b=0$. \bigskip

3. Ici, ce qui compte, ce sont les naissances futures, qui \'{e}largissent
la base fiscale. La probabilit\'{e} de mort laisse intacte la contrainte budg%
\'{e}taire intertemporelle des agents : car si je ne meurs pas d'ici demain,
la mort ayant frapp\'{e} d'autres que moi, le fardeau fiscal par t\^{e}te
est plus \'{e}lev\'{e}. En revanche, quand l'horizon des agents est fini de
mani\`{e}re certaine, l'argument reste valable.\bigskip \bigskip

\textbf{Bouclage du mod\`{e}le :\medskip }

Offre de travail in\'{e}lastique et technologie transformant une unit\'{e}
de travail en une unit\'{e} de bien, de sorte que $w_{t}=1$ $\forall t$. Le
march\'{e} des biens est \'{e}quilibr\'{e} lorsque $C_{t}+G_{t}=N_{t}$.
\bigskip

Un \'{e}quilibre, c'est un taux $r$, et des quantit\'{e}s $\{c_{s,t}\}$, $%
\{a_{s,t}\}$, $\{G_{t}\}$, $\{T_{t}\}$, et $\{D_{t}\}$\ qui v\'{e}rifient
:\medskip

-- optimisation individuelle :

\begin{equation*}
c_{s,t}=\Omega (a_{s,t}+h_{t})
\end{equation*}

-- \'{e}quilibre au niveau agr\'{e}g\'{e} sur le march\'{e} des biens et sur
les titres \'{e}mis par l'Etat%
\begin{equation*}
C_{t}+G_{t}=N_{t}
\end{equation*}

\begin{equation*}
A_{t}=D_{t}
\end{equation*}

-- absence de d\'{e}rive des finances publiques

\begin{equation*}
\sum_{i=0}^{\infty }\frac{T_{t+i}}{(1+r)^{i}}=D_{t}+\sum_{i=0}^{\infty }%
\frac{G_{t+i}}{(1+r)^{i}}
\end{equation*}%
\newpage

\bigskip

\textbf{Mise en perspective de la litt\'{e}rature :}\bigskip

-- Robert Barro, "Are Government Bonds Net Wealth?", JPE 1974

cas o\`{u} $p=b=n=0$.

Il y a \'{e}quivalence ricardienne.\bigskip

-- Olivier Blanchard, "Debt, Deficits and Finite Horizons", JPE 1985

cas o\`{u} $p=b>0$ et $n=0$.

Non-neutralit\'{e} de la dette.

Erreur d'interpr\'{e}tation : croire que la non-neutralit\'{e} vient de $p>0$%
.\bigskip

-- Philippe Weil, Harvard PhD Thesis, 1985

cas o\`{u} $p=0$, $b=n>0$.

Non-neutralit\'{e} de la dette.

\bigskip

\bigskip

Tous ces papiers supposent que la consommation est correctement pr\'{e}dite
par la th\'{e}orie du revenu permanent. N\'{e}anmoins, \textbf{la pr\'{e}%
sence de contraintes de liquidit\'{e}} est une bonne raison de penser que le
mode de financement des d\'{e}ficits n'est pas neutre, ind\'{e}pendamment
des consid\'{e}rations d\'{e}mographiques. \textbf{\medskip }

Si les contraintes de liquidit\'{e} mordent pour certains agents (i.e. leur
niveau de consommation optimal -- celui pr\'{e}dit par la th\'{e}orie du
revenu permanent -- est plus \'{e}lev\'{e} que ce que le niveau de liquidit%
\'{e} dont ils disposent), ils utilisent l'augmentation de leur revenu
disponible (suite au choix fait par l'Etat d'\'{e}mettre de la dette plut%
\^{o}t que de pr\'{e}lever des imp\^{o}ts) pour consommer plus et non pas
pour \'{e}pargner en vue des imp\^{o}ts futurs.

\end{document}
