% partie d�clarative
\documentclass[a4paper,10pt]{article}
%\pdfoptionpdfminorversion = 5

\usepackage[english]{babel} %,francais
%\usepackage[latin1]{inputenc}
%\usepackage[T1]{fontenc}\usepackage{amsfonts}
\usepackage{amssymb}
\usepackage{amsmath}
\usepackage{amsthm}
\usepackage{bbm}
\usepackage{stmaryrd}
\usepackage{enumerate}
%usepackage{url}
%usepackage{wasysym}
\usepackage[colorlinks=true,urlcolor=blue,pdfstartview=FitH]{hyperref} 
%usepackage{lscape}
%usepackage{manfnt}
%usepackage{mathbbol}
%usepackage{eurosym}
\usepackage{color}
%\usepackage{bbding}
\usepackage[cyr]{aeguill}     % Police vectorielle TrueType, guillemets fran�ais
\usepackage[pdftex]{graphicx} % Pour l'insertion d'images


%\usepackage{aertt}
\usepackage{harvard}
\usepackage{cje}

%\usepackage[cyr]{aeguill}     % Police vectorielle TrueType, guillemets fran�ais
\usepackage[pdftex]{graphicx} % Pour l'insertion d'images
\DeclareGraphicsExtensions{.jpg,.mps,.pdf,.png,.bmp}%TCIDATA{OutputFilter=LATEX.DLL}
%TCIDATA{Version=4.10.0.2345}
%TCIDATA{Created=Thursday, August 25, 2005 14~:35~:37}
%TCIDATA{LastRevised=Tuesday, October 18, 2005 15~:55~:08}
%TCIDATA{<META NAME="GraphicsSave" CONTENT="32">}
%TCIDATA{<META NAME="DocumentShell" CONTENT="Standard LaTeX\Blank - Standard LaTeX Article">}
%TCIDATA{CSTFile=40 LaTeX article.cst}

%\newtheorem{theorem}{Theorem}
%\newtheorem{acknowledgement}[theorem]{Acknowledgement}
%\newtheorem{algorithm}[theorem]{Algorithm}
%\newtheorem{axiom}[theorem]{Axiom}
%\newtheorem{case}[theorem]{Case}
%\newtheorem{claim}[theorem]{Claim}
%\newtheorem{conclusion}[theorem]{Conclusion}
%\newtheorem{condition}[theorem]{Condition}
%\newtheorem{conjecture}[theorem]{Conjecture}
%\newtheorem{corollary}[theorem]{Corollary}
%\newtheorem{criterion}[theorem]{Criterion}
%\newtheorem{definition}[theorem]{Definition}
%\newtheorem{example}[theorem]{Example}
%\newtheorem{exercise}[theorem]{Exercise}
%\newtheorem{lemma}[theorem]{Lemma}
%\newtheorem{notation}[theorem]{Notation}
%\newtheorem{problem}[theorem]{Problem}
%\newtheorem{proposition}[theorem]{Proposition}
%\newtheorem{remark}[theorem]{Remark}
%\newtheorem{solution}[theorem]{Solution}
%\newtheorem{summary}[theorem]{Summary}
%\newenvironment{proof}[1][Proof]{\noindent\textbf{#1.} }{\ \rule{0.5em}{0.5em}}
%\input{tcilatex}

\newcommand{\EnTete}[3]{
		\begin{flushleft}
			Fran\c{c}ois Le Grand \hfill #1
			
			legrand@pse.ens.fr
			
			Cours de Macro�conomie 4 (Prof. Daniel Cohen)
		\end{flushleft}
		\begin{center}	
			\bigskip
			TD {#2}
			
			\bigskip
			\textbf{#3} 
		\end{center}

\hrule\vspace{\baselineskip} 
}
%\noindent
\renewcommand{\thesection}{\Alph{section}}
\setlength{\parindent}{0pt}

\newcounter{question}
\newcommand{\thenumq}{\arabic{question}}
\newcommand{\ques}{\addtocounter{question}{1}{\hskip.4cm{\thenumq.~\,}}}
\newcommand{\textenitalique}[1]{
\bigskip
\textit{#1} \\\nopagebreak
}


\begin{document}

\EnTete{S�ances des 3 et 6 Octobre 2006}{1}{Mod\`{e}les dynamiques de \textquotedblleft job-search\textquotedblright\ \`{a} la Mortensen-Pissarides}

\bibliographystyle{cje}
\bibliography{Biblio_TD}
\nocite{Ho:90}\nocite{MoPi:94}\nocite{MoPi:98}\nocite{Pi:90}

\paragraph{\large{Points techniques du TD~:}} l'optimisation sous contraintes~:
\begin{itemize}
	\item Lagrangien
	\item Hamiltonien
	\item \'Equation de Bellman
\end{itemize}

\section{Description du mod�le}
On consid\`{e}re le mod\`{e}le standard \`{a} la Mortensen-Pissarides, o� les travailleurs et firmes sont neutres au risque. 

\textenitalique{Les travailleurs}

Les travailleurs maximisent  la fonction suivante o\`{u} $r$ d\'{e}signe le taux d'escompte et $y(s)$ le revenu net \`{a} la p\'{e}riode $s$~:
\begin{equation*}
V_{t}=\int_{t}^{\infty }e^{-rs}y(s)ds
\end{equation*}

Les travailleurs employ\'{e}s sont r\'{e}mun\'{e}r\'{e}s au salaire $w$. Un
travailleur au ch\^{o}mage touche le salaire de r\'{e}serve $b$ ($b$ peut 
\^{e}tre interpr\'{e}t\'{e} comme une allocation ch\^{o}mage ou comme le
gain du loisir).

\textenitalique{Les firmes}

Il existe un large nombre d'entreprises, toutes identiques (libre-entr\'{e}e
de firmes). Une firme qui emploie un travailleur produit $x$ ($\equiv$ productivit\'{e} du travail).


\textenitalique{ Matching function }

On d\'{e}signe par $v$ le nombre de postes vacants dans les entreprises. Le
co\^{u}t \`{a} chaque p\'{e}riode pour une firme qui cherche \`{a} remplir
un poste vacant est $\gamma $.

On normalise \`{a} 1 la taille de la population active, de sorte que $u$\ d%
\'{e}signe \`{a} la fois le taux de ch\^{o}mage et le nombre de ch\^{o}meurs.

A chaque p\'{e}riode, un travailleur employ\'{e} a une probabilit\'{e}
instantan\'{e}e $s$ constante de se retrouver au ch\^{o}mage.

On suppose que le nombre d'embauches \`{a} une p\'{e}riode donn\'{e}e\ est
une fonction du nombre de personnes au ch\^{o}mage $\left( u\right) $\ et du
nombre de postes vacants $\left( v\right) $~:%
\begin{equation*}
m(u,v)
\end{equation*}

On supposera que\ la fonction $m$\ est \`{a} rendements constants, du type~: 
$m(u,v)=u^{\alpha }v^{1-\alpha }$.

\section{R�solution}

\ques Exprimer la probabilit\'{e} instantan\'{e}e $h$ pour un ch\^{o}meur de trouver un emploi.

\medskip

\ques Montrer que $h$\ peut s'\'{e}crire $\theta q(\theta )$, avec $\theta
\equiv \frac{v}{u}$ et $q$ une fonction d\'{e}croissante. Que repr\'{e}sente 
$q$~? Les travailleurs ont-ils int\'{e}r\^{e}t \`{a} ce que $\theta $\ soit 
\'{e}lev\'{e} ou faible~? Exprimer la dur\'{e}e moyenne de vacance d'un poste
en fonction de $q(\theta )$.

\medskip

\ques Calculer le taux de ch\^{o}mage \`{a} l'\'{e}quilibre stationnaire en
fonction de~$\left\{ s,h\right\} $.

\medskip

Pour toutes, les questions suivantes, on se place \`{a} l'\'{e}tat
stationnaire.

On note $J_{F}$ (resp. $J_{V}$) la valeur actualis\'{e}e des revenus pour
une firme d'une place occup\'{e}e par un travailleur (resp. la valeur
actualis\'{e}e des revenus pour une firme d'une place vacante).

On note $J_{E}$ (resp. $J_{U}$) la valeur actualis\'{e}e des revenus pour un
travailleur employ\'{e} (resp. la valeur actualis\'{e}e des revenus pour un
travailleur en recherche d'emploi).

\medskip

\ques Expliquer pourquoi \`{a} l'\'{e}quilibre stationnaire, $J_{F}$ et $J_{V}$
v\'{e}rifient~:

\begin{equation*}
rJ_{F}=x-w+s(J_{V}-J_{F})
\end{equation*}

On utlisera trois m�thodes diff�rentes~: (i) fonction valeur sous forme int�grale (ii) sous forme diff�rentielle (iii) un raisonnement d'arbitrage.

Exprimer de m\^{e}me $J_{V}$ en fonction $J_{F}$ et des param\`{e}tres du mod\`{e}les.

\medskip

\ques Interpr\'{e}ter la condition d'\'{e}quilibre suivante~:%
\begin{equation*}
\frac{\gamma }{q(\theta )}=\frac{x-w}{r+s}
\end{equation*}

\medskip

\ques Exprimer $J_{E}$ en fonction de $J_{U}$ et des param\`{e}tres du mod\`{e}%
les (resp. $J_{U}$ en fonction de $J_{E}$ et des param\`{e}tres du mod\`{e}%
les).

\textenitalique{Nash-Bargaining et \'{e}quation de salaire}

On suppose que le salaire est d\'{e}termin\'{e} de sorte que la fonction
suivante soit maximis\'{e}e~:%
\begin{equation*}
\max_{\left( w\right) }J_{F}{}^{\beta }(J_{E}-J_{U})^{1-\beta }
\end{equation*}

o\`{u} $\beta $ d\'{e}signe le pouvoir de n\'{e}gociation de la firme.

On notera que le salaire d'\'{e}quilibre impacte $J_{U}$ mais que cet effet
n'est pas internalis\'{e} par le travailleur qui n\'{e}gocie son salaire
avec une firme donn\'{e}e (prenant le salaire d'\'{e}quilibre comme donn\'{e}%
).

\medskip

\ques Montrer que~: $\beta \left( J_{E}-J_{U}\right) =\left( 1-\beta \right)
J_{F}$.

Montrer qu'\`{a} l'\'{e}quilibre, $w=\beta b+(1-\beta )(x+\gamma \theta )$.
Commenter.

\textenitalique{Equilibre sur le march\'{e} du travail et statique comparative}

\ques Repr\'{e}senter graphiquement la d\'{e}termination de $(\theta ,w)$ dans
le plan $(\theta ,w)$.

D\'{e}terminer comment $\theta $, $h$ et $u$ sont affect\'{e}s par~:

\begin{itemize}
	\item une hausse de $b$,
	\item une hausse de $\gamma $,
	\item  une baisse de $s$.
\end{itemize}

\section{Efficience des mod\`{e}les dynamiques de search~: Condition
d'Hosios}

\medskip

On cherche \`{a} d\'{e}terminer l'allocation optimale de l'emploi et \`{a}
la comparer \`{a} l'\'{e}quilibre d\'{e}centralis\'{e}. Le planificateur
cherche \`{a} maximiser la production sous la contrainte d'\'{e}volution de
l'emploi.

\medskip

\ques Exprimer le surplus $\left( S\right) $ du planificateur \`{a} chaque p%
\'{e}riode avec $u$ chomeurs et $v$ postes vacants.

\medskip

\ques Ecrire le programme du planificateur sous contrainte. Ecrire le
Hamiltonien associ\'{e}. En d\'{e}duire les conditions du premier ordre.

On introduira $\eta (\theta )=-\frac{\frac{dq}{d\theta }\theta }{q(\theta )}$
(Indication~: il est conseill\'{e} de prendre $\theta =\frac{v}{u}$ comme
variable de contr\^{o}le plut\^{o}t que $v$).

\medskip

\ques Montrer qu'\`{a} l'\'{e}quilibre stationnaire on a~:

\begin{equation*}
(1-\eta (\theta ))(x-b)-\gamma \frac{r+s+\eta (\theta )\theta\,q(\theta )}{q(\theta )}=0
\end{equation*}

\medskip

\ques En utilisant les questions 4) et 6), montrer que dans le cas de l'\'{e}%
quilibre d\'{e}centralis\'{e}, on a la relation suivante~:

\begin{equation*}
\beta (x-b)-\gamma \frac{r+s+\left( 1-\beta \right) \theta q(\theta )}{q(\theta )}=0
\end{equation*}

\medskip

\ques En d\'{e}duire \`{a} quelle condition l'\'{e}quilibre d\'{e}centralis%
\'{e} est efficient. C'est la {\bf Condition d'Hosios}. Comment s'\'{e}crit cette condition
lorsque la fonction de matching est Cobb--Douglas. Commenter.


\end{document}
