
\documentclass[a4paper]{article}
%%%%%%%%%%%%%%%%%%%%%%%%%%%%%%%%%%%%%%%%%%%%%%%%%%%%%%%%%%%%%%%%%%%%%%%%%%%%%%%%%%%%%%%%%%%%%%%%%%%%%%%%%%%%%%%%%%%%%%%%%%%%
\usepackage{amsmath}

\setcounter{MaxMatrixCols}{10}
%TCIDATA{OutputFilter=LATEX.DLL}
%TCIDATA{Version=4.10.0.2345}
%TCIDATA{Created=Thursday, August 25, 2005 14:35:37}
%TCIDATA{LastRevised=Monday, October 31, 2005 11:04:12}
%TCIDATA{<META NAME="GraphicsSave" CONTENT="32">}
%TCIDATA{<META NAME="DocumentShell" CONTENT="Standard LaTeX\Blank - Standard LaTeX Article">}
%TCIDATA{CSTFile=40 LaTeX article.cst}

\newtheorem{theorem}{Theorem}
\newtheorem{acknowledgement}[theorem]{Acknowledgement}
\newtheorem{algorithm}[theorem]{Algorithm}
\newtheorem{axiom}[theorem]{Axiom}
\newtheorem{case}[theorem]{Case}
\newtheorem{claim}[theorem]{Claim}
\newtheorem{conclusion}[theorem]{Conclusion}
\newtheorem{condition}[theorem]{Condition}
\newtheorem{conjecture}[theorem]{Conjecture}
\newtheorem{corollary}[theorem]{Corollary}
\newtheorem{criterion}[theorem]{Criterion}
\newtheorem{definition}[theorem]{Definition}
\newtheorem{example}[theorem]{Example}
\newtheorem{exercise}[theorem]{Exercise}
\newtheorem{lemma}[theorem]{Lemma}
\newtheorem{notation}[theorem]{Notation}
\newtheorem{problem}[theorem]{Problem}
\newtheorem{proposition}[theorem]{Proposition}
\newtheorem{remark}[theorem]{Remark}
\newtheorem{solution}[theorem]{Solution}
\newtheorem{summary}[theorem]{Summary}
\newenvironment{proof}[1][Proof]{\noindent\textbf{#1.} }{\ \rule{0.5em}{0.5em}}
\input{tcilatex}

\newcommand{\correction}[1]{
\begin{changemargin}{1cm}{0cm}
   \it #1
\end{changemargin} }

\begin{document}


{\small Nicolas Coeurdacier}

{\small nicolas.coeurdacier@pse.ens.fr}

{\small Cours de Macro\'{e}conomie 3 (Prof. D.\ Cohen)}

\begin{center}
\bigskip

TD 2 - El\'{e}ments de correction

\textbf{Architecture Optimale de la Protection de l'Emploi}
\end{center}

\bigskip

\textit{R\'{e}f\'{e}rences}: Blanchard, O. et Tirole, J., 2004, "The Optimal
Design of Unemployment Insurance and Employment Protection : A first pass", 
\textit{mimeo MIT et IDEI}\footnote{%
http://idei.fr/doc/by/tirole/optimal.pdf}\textit{.}

\bigskip

\textbf{Set-up du mod\`{e}le}

\medskip

\textit{Hypoth\`{e}ses}

\begin{itemize}
\item L'\'{e}conomie est constitu\'{e}e d'un continuum de travailleurs de
masse $1$ et d'un continuum d'entrepreneurs de masse $1$.

\item Les entrepreneurs sont neutres au risque. Chaque entrepreneur peut
mener un projet (cr\'{e}er une entreprise). Il y a un co\^{u}t fixe au
lancement d'un projet $I$, identique pour tous les entrepreneurs. Si un
projet est lanc\'{e}, un travailleur est employ\'{e} et la productivit\'{e}
du \textquotedblleft match\textquotedblright\ entrepreneur-travailleur $(y)$
est r\'{e}v\'{e}l\'{e}e. La productivit\'{e} $(y)$ d'un match est tir\'{e}e
d'une fonction de distribution dont la fonction de r\'{e}partition est $G(y)$
continue et diff\'{e}rentiable (on note $g(y)$ dans $[0,1]$ la densit\'{e}
associ\'{e}e). Une fois la productivit\'{e} r\'{e}v\'{e}l\'{e}e,
l'entreprise peut soit produire et r\'{e}mun\'{e}rer le travailleur, soit le
licencier (lequel se retrouve au ch\^{o}mage).

\item Les travailleurs sont averses au risque: leur fonction d'utilit\'{e}
est $U(x)$ concave, o\`{u} $x$ est le revenu du travailleur. En l'absence
d'allocation ch\^{o}mage, le revenu de r\'{e}serve d'un travailleur sans
emploi est $b$.

\item Le dernier agent de cet \'{e}conomie est l'Etat: l'Etat finance les
allocations-ch\^{o}mage $(\mu )$ \`{a} budget \'{e}quilibr\'{e}.\ Pour
financer la caisse d'allocations, il peut avoir recours \`{a} deux
instruments: taxe au licenciement d'un travailleur $\left( f\right) $ pay%
\'{e}es \`{a} chaque licenciement ou cotisations sociales $\left( \tau
\right) $ pay\'{e}es par la firme pour chaque travailleur employ\'{e}.
\end{itemize}

\textit{Timing}

\begin{itemize}
\item P\'{e}riode 0: L'Etat choisit le triplet $\left\{ f,\tau ,\mu \right\} 
$ sous-contrainte de budget \'{e}quilibr\'{e}.

\item P\'{e}riode 1: Les entrepreneurs d\'{e}cident de lancer un projet,
paient le co\^{u}t fixe $(I)$ et emploient des travailleurs.\ Les
entrepreneurs offrent un contrat $\left( w,y^{\ast }\right) $ qui stipule un
niveau de salaire fix\'{e} \textit{ex-ante} $w$ et un niveau minimum de
productivit\'{e} $(y^{\ast })$ en de\c{c}a duquel le travailleur est licenci%
\'{e}. Noter que les firmes ne proposent pas de salaires contingents \`{a}
la r\'{e}alisation $(w(y))$ du fait de l'aversion au risque des
travailleurs. Noter aussi que comme les entrepreneurs font face au m\^{e}me
co\^{u}t fixe et \`{a} la m\^{e}me distribution de productivit\'{e}, \`{a} l'%
\'{e}quilibre, ils chercheront tous \`{a} lancer un projet.

\item P\'{e}riode 2: L'incertitude sur la productivit\'{e} est r\'{e}v\'{e}l%
\'{e}e: si $y>y^{\ast }$, la production est r\'{e}alis\'{e}e et le salaire $%
\left( w\right) $ est pay\'{e} au travailleur et la cotisation sociale $%
\left( \tau \right) $ est pay\'{e} \`{a} l'Etat; si $y<y^{\ast }$, la firme
paie la taxe $\left( f\right) $ et le travailleur est licenci\'{e} et touche 
$\left( \mu \right) $ de l'Etat.
\end{itemize}

\bigskip

L'objet du TD est de trouver l'architecture optimale de la protection de
l'emploi, \textit{i.e }le triplet $\left\{ f,\tau ,\mu \right\} $ optimal
dans divers cas de figure.

\bigskip

\textbf{A.\ Benchmark}

\bigskip

1) Ecrire l'utilit\'{e} esp\'{e}r\'{e}e d'un travailleur $(V_{W})$
conditionnellement au contrat $\left( w,y^{\ast }\right) $.

\begin{equation*}
V_{W}=G(y^{\ast })U(b+\mu )+\left( 1-G(y^{\ast })\right) U(w)
\end{equation*}

\bigskip

2) Ecrire le gain esp\'{e}r\'{e} d'un entrepreneur $(V_{F})$
conditionnellement au contrat $\left( w,y^{\ast }\right) $? A quelle
condition, les entrepreneurs lancent un projet?%
\begin{equation*}
V_{F}=-G(y^{\ast })f+\int_{y^{\ast }}^{\infty }(y-(w+\tau ))dG(y)=-G(y^{\ast
})f+\int_{y^{\ast }}^{\infty }ydG(y)-\left( 1-G(y^{\ast })\right) (w+\tau )
\end{equation*}

Contrainte de participation (ou \'{e}quation de libre entr\'{e}e):%
\begin{equation*}
V_{F}\geq I
\end{equation*}

\bigskip

3) Ecrire la contrainte budg\'{e}taire du gouvernement.%
\begin{equation*}
G(y^{\ast })\mu =G(y^{\ast })f+\left( 1-G(y^{\ast })\right) \tau
\end{equation*}

\bigskip

\textit{Programme de la firme}

\bigskip

4) Ecrire le programme de maximisation de la firme (N'oubliez pas la
contrainte de participation des travailleurs!). Donner les conditions du
premier-ordre. En d\'{e}duire que:%
\begin{equation*}
y^{\ast }=w+\tau -f-\left( \frac{U(w)-U(b+\mu )}{U^{\prime }(w)}\right)
\end{equation*}

Commenter.%
\begin{eqnarray*}
&&\max_{\left\{ w,y^{\ast }\right\} }V_{F} \\
s.c &:&V_{W}\geq \overline{U}
\end{eqnarray*}

La firme propose le contrat qui maximise son profit sous contrainte que les
travailleurs acceptent de participer au march\'{e} du travail (en gagnant au
moins leur utilit\'{e} de r\'{e}serve $\overline{U}$). En introduisant $%
\lambda _{1}$ le multiplicateur de Lagrange de la contrainte de
participation des travailleurs, on a les CPO suivantes:%
\begin{eqnarray*}
/y^{\ast }\text{ \ \ \ \ \ \ \ }0 &=&\frac{\partial V_{F}}{\partial y^{\ast }%
}+\lambda _{1}\frac{\partial V_{W}}{\partial y^{\ast }}=g(y^{\ast })(w+\tau
-f-y^{\ast })+\lambda _{1}g(y^{\ast })\left( U(b+\mu )-U(w)\right)  \\
/w\text{ \ \ \ \ \ \ \ }0 &=&\frac{\partial V_{F}}{\partial w}+\lambda _{1}%
\frac{\partial V_{W}}{\partial w}=-\left( 1-G(y^{\ast })\right) +\lambda
_{1}\left( 1-G(y^{\ast })\right) U^{\prime }(w)
\end{eqnarray*}%
\begin{eqnarray*}
&\Rightarrow &\lambda _{1}=\frac{1}{U^{\prime }(w)} \\
\text{donc} &\text{:}&w+\tau -f-y^{\ast }=-\frac{U(b+\mu )-U(w)}{U^{\prime
}(w)}
\end{eqnarray*}

soit:%
\begin{equation*}
y^{\ast }=w+\tau -f-\frac{U(w)-U(b+\mu )}{U^{\prime }(w)}
\end{equation*}

5) Pourquoi cette solution n'est coh\'{e}rente temporellement? Donner la
solution au programme pr\'{e}c\'{e}dent coh\'{e}rente temporellement.

\bigskip

\textit{Ex-ante}, la valeur seuil de productivit\'{e} pour laquelle
l'entreprise licencie le travailleur est \'{e}gale au co\^{u}t net de
production $\left( w+\tau -f\right) $ moins le terme $\frac{U(b+\mu )-U(w)}{%
U^{\prime }(w)}$. A moins que le travailleur soit parfaitement assur\'{e} $%
(w=b+\mu )$, ce terme est est positif: le niveau seuil est donc inf\'{e}%
rieur au co\^{u}t net de production. La firme neutre au risque s'engage 
\textit{ex-ante} \`{a} assurer le travailleur averse au risque.\ Pourtant,
ce contrat n'est pas coh\'{e}rent temporellement: en effet, \textit{ex-post}%
, le seuil pour lequel la firme est indiff\'{e}rente entre licencier le
travailleur et le r\'{e}mun\'{e}rer est tel que $-f=y^{\ast }-\left( w+\tau
\right) $ soit $y^{\ast }=w+\tau -f$.

Cette solution est coh\'{e}rente intertemporellement.

\bigskip

On supposera par la suite que le contrat $\left( w,y^{\ast }\right) $ est coh%
\'{e}rent temporellement.

\bigskip

\textit{Optimum social}

\bigskip

Un planificateur bienveillant cherche \`{a} maximiser l'utilit\'{e} esp\'{e}r%
\'{e}e des travailleurs sous les contraintes suivantes:

- contrainte budg\'{e}taire du gouvernement

- contrainte de participation des entrepreneurs

- seuil de productivit\'{e} choisi par les entreprises (et coh\'{e}rent
temporellement)

\bigskip

6) Montrer que les contraintes de participation des entrepreneurs et d'\'{e}%
quilibre budg\'{e}taire implique la contrainte suivante:

\begin{equation*}
-G(y^{\ast })\mu +\int_{y^{\ast }}^{\infty }ydG(y)-\left( 1-G(y^{\ast
})\right) w\geq I
\end{equation*}

On a:

\begin{equation*}
-G(y^{\ast })f-\left( 1-G(y^{\ast })\right) \tau +\int_{y^{\ast }}^{\infty
}ydG(y)-\left( 1-G(y^{\ast })\right) w\geq I
\end{equation*}

\begin{equation*}
G(y^{\ast })\mu =G(y^{\ast })f+\left( 1-G(y^{\ast })\right) \tau
\end{equation*}

d'o\`{u}:

\begin{equation*}
-G(y^{\ast })\mu +\int_{y^{\ast }}^{\infty }ydG(y)-\left( 1-G(y^{\ast
})\right) w\geq I
\end{equation*}

Du fait de la contrainte de budget, la fa\c{c}on dont sont financ\'{e}es les
allocations ch\^{o}mage ne modifie pas la contrainte de participation des
entrepreneurs, seul compte le niveau d'allocations vers\'{e}es.

\bigskip

On suppose que le planificateur choisit d'abord $\{w,\mu ,y^{\ast }\}$ de
sorte \`{a} maximiser:

\begin{eqnarray*}
&&\max_{\{w,\mu ,y^{\ast }\}}\left\{ G(y^{\ast })U(b+\mu )+\left(
1-G(y^{\ast })\right) U(w)\right\} \\
s.c &:&\text{ }-G(y^{\ast })\mu +\int_{y^{\ast }}^{\infty }ydG(y)-\left(
1-G(y^{\ast })\right) w\geq I
\end{eqnarray*}

(En effet, le planificateur pourra toujours fixer le couple $\{f,\tau \}$
tel que $y^{\ast }$ corresponde au seuil choisi par les entreprises et tel
que la contrainte budg\'{e}taire de l'Etat est \'{e}quilibr\'{e}e).

En d\'{e}duire le niveau d'allocations-ch\^{o}mage vers\'{e}es par le
gouvernement et le seuil de productivit\'{e} \`{a} l'optimum social en
fonction de $w$ et $b$. Commenter.

\bigskip

Soit $\lambda _{2}$ le multiplicateur de Lagrange associ\'{e} \`{a} ce
programme:

CPO:%
\begin{eqnarray*}
&&\max_{\{w,\mu ,y^{\ast }\}}\left\{ G(y^{\ast })U(b+\mu )+\left(
1-G(y^{\ast })\right) U(w)\right\} \\
s.c &:&\text{ }-G(y^{\ast })\mu +\int_{y^{\ast }}^{\infty }ydG(y)-\left(
1-G(y^{\ast })\right) w\geq I
\end{eqnarray*}%
\begin{eqnarray*}
/y^{\ast }\text{ \ \ \ \ \ \ \ }g(y^{\ast })\left( U(b+\mu )-U(w)\right)
+\lambda _{2}g(y^{\ast })\left[ -\mu -y^{\ast }+w\right] &=&0 \\
/w\text{ \ \ \ \ \ \ \ \ \ \ \ \ \ \ \ \ \ \ \ \ \ \ \ \ \ }\left(
1-G(y^{\ast })\right) U^{^{\prime }}(w)-\lambda _{2}\left( 1-G(y^{\ast
})\right) &=&0 \\
/\mu \text{ \ \ \ \ \ \ \ \ \ \ \ \ \ \ \ \ \ \ \ \ \ \ \ \ \ \ \ \ \ \ \ \
\ \ \ \ \ \ }G(y^{\ast })U^{^{\prime }}(b+\mu )-\lambda _{2}G(y^{\ast }) &=&0
\end{eqnarray*}%
\begin{eqnarray*}
U(b+\mu )-U(w)+\lambda _{2}\left[ -\mu -y^{\ast }+w\right] &=&0 \\
U^{^{\prime }}(w) &=&\lambda _{2} \\
U^{^{\prime }}(b+\mu ) &=&\lambda _{2}
\end{eqnarray*}

soit%
\begin{eqnarray*}
w &=&b+\mu \\
y^{\ast } &=&b
\end{eqnarray*}

La premi\`{e}re condition est celle d'assurance compl\`{e}te: les
travailleurs ont le m\^{e}me niveau d'utilit\'{e} qu'ils soient employ\'{e}s
ou non. Puisque les assurer parfaitement n'induit pas les travailleurs \`{a}
faire moins d'effort (pas d'al\'{e}a moral, cf. section B), d\`{e}s lors
qu'ils sont averses au risque, l'optimum correspond au cas d'assurance compl%
\`{e}te.

La deuxi\`{e}me condition est une condition d'efficacit\'{e}: il est
efficace que les firmes produisent tant que produire rapporte plus que l'%
\'{e}quivalent salaire de ne pas travailler.\ Autrement dit, \`{a} l'optimum
social, la firme doit renvoyer le travailleur si il est capable de faire
mieux en \'{e}tant non-employ\'{e}!

\bigskip

7) En d\'{e}duire le design $\{f,\tau \}$ choisi par le gouvernement pour r%
\'{e}aliser l'optimum social. Commenter.

\bigskip

Le $y^{\ast }$ choisi par la firme \`{a} l'\'{e}quilibre v\'{e}rifie: $%
y^{\ast }=w+\tau -f$

Donc, pour \^{e}tre \`{a} l'optimum, il faut:%
\begin{equation*}
\mu =f-\tau
\end{equation*}

Contrairement \`{a} l'intution, les taxes aux licenciements et les
cotisations sociales sont compl\'{e}mentaires (et non substituts) pour \^{e}%
tre efficaces. A allocations ch\^{o}mages donn\'{e}es, si $\tau $ augmente $%
f $ doit augmenter: la raison est simple, si $\tau $ augmente, les firmes
sont plus incit\'{e}es \`{a} licencier les travailleurs car il faut cotiser
pour chaque travailleur en place $(y^{\ast }\uparrow )$, pour \'{e}viter
qu'elles ne le fassent on augmente la taxe de licenciement.

\bigskip

Enfin, l'\'{e}quilibre budg\'{e}taire de l'Etat nous donne le couple optimal
(avec $y^{\ast }=b$):%
\begin{eqnarray*}
G(b)\mu &=&G(b)f+\left( 1-G(b)\right) \tau \\
&=&\tau +G(b)\left( f-\tau \right) \\
&=&\tau +G(b)\mu \\
&\Rightarrow &\tau =0
\end{eqnarray*}

Donc le couple $\{f,\tau \}$ optimal est $\{\mu ,0\}$. Les allocations ch%
\^{o}mages sont \`{a} l'optimum uniquement financ\'{e} par une taxe au
licenciement. En effet, cette taxe permet aux entrepreneurs d'internaliser
parfaitement l'externalit\'{e} n\'{e}gative induit par les licenciements sur
l'\'{e}conomie.\ Efficacit\'{e} productive et parfaite assurance des
travailleurs sont rendues possibles par cette taxe. Pourquoi $\tau >0$ est
sous-optimal? Augmenter $\tau $ augmente les licenciements (en augmentant le
co\^{u}t du travail), afin de maintenir l'efficacit\'{e} productive, il faut
donc augmenter simultan\'{e}ment $f$: cela augmente les revenus du
gouvernement (mais le nombre de ch\^{o}meurs est identique), le gouvernement
en tire un surplus inutile au d\'{e}triment des salaires des travailleurs
(cf. \'{e}quation question 6)).

\bigskip

\textbf{B.\ Limites de l'assurance compl\`{e}te: al\'{e}a moral}

\bigskip

Dans le mod\`{e}le benchmark, les travailleurs sont parfaitement assur\'{e}%
s: en pratique, ce cas est peu r\'{e}aliste car une parfaite assurance d\'{e}%
sincite les travailleurs \`{a} fournir l'effort maximal pour chercher un
emploi et/ou garder leur emploi (al\'{e}a moral). Nous allons voir dans
cette deuxi\`{e}me partie comment ces questions d'incitations modifient le
design de la protection de l'emploi.

\bigskip \pagebreak

Nous supposons que la contrainte d'incitation des travailleurs (\`{a}
fournir un effort lorsqu'ils sont employ\'{e}s) peut se r\'{e}\'{e}crire de
la mani\`{e}re suivante:%
\begin{equation*}
\left( 1-G(y^{\ast })\right) \left( U(w)-U(b+\mu )\right) \geq B\text{ \ \ \
\ \ \ \ \ \ \ \ \ \ \ \ \ \ \ \ }(IW)
\end{equation*}

o\`{u} $B$ d\'{e}signe les b\'{e}n\'{e}fices priv\'{e}s d'un travailleur
lorsque celui -ci ne fournit pas d'effort.

\bigskip

8) Interpr\'{e}ter la condition $(IW)$

\bigskip

Une interpr\'{e}tation simple de cette condition est la suivante: lorsque un
travailleur \`{a} un emploi, il peut d\'{e}cider ne fournir un effort ou de
ne pas fournir d'effort: ne pas fournir d'effort lui conf\`{e}re le gain $B$
en terme d'utilit\'{e} (il est alors licenci\'{e} et rien n'est produit).
Comme l'effort n'est pas observable (et donc pas \textquotedblleft
contractible\textquotedblright ), la contrainte pr\'{e}c\'{e}dente doit \^{e}%
tre v\'{e}rifi\'{e}e pour que le travailleur fournisse un effort.%
\begin{equation*}
V_{W}\geq B+U(b+\mu )=\text{utilit\'{e} en l'absence d'effort}
\end{equation*}

Une autre interpr\'{e}tation possible est que $B$ est le co\^{u}t \`{a}
l'effort de recherche d'emploi (effort non-observable). Si $V_{W}\leq
B+U(b+\mu )$, le travailleur pr\'{e}f\`{e}re rester au ch\^{o}mage et ne pas
fournir d'effort pour trouver un emploi.

\bigskip

\textit{Optimum social}

\bigskip

9) R\'{e}\'{e}crire le programme du planificateur social. Montrer que

a) l'assurance n'est plus compl\`{e}te

b) la valeur seuil v\'{e}rifie $y^{\ast }=b+\left[ w-(b+\mu )-\frac{%
U(w)-U(b+\mu )}{U^{\prime }(w)}\right] $

Commenter.%
\begin{eqnarray*}
&&\max_{\{w,\mu ,y^{\ast }\}}\left\{ G(y^{\ast })U(b+\mu )+\left(
1-G(y^{\ast })\right) U(w)\right\} \\
s.c &:&\text{ }-G(y^{\ast })\mu +\int_{y^{\ast }}^{\infty }ydG(y)-\left(
1-G(y^{\ast })\right) w\geq I \\
s.c &:&\left( 1-G(y^{\ast })\right) \left( U(w)-U(b+\mu )\right) \geq B
\end{eqnarray*}

a) est trivial d'apr\`{e}s la troisi\`{e}me contrainte: $\left( U(w)-U(b+\mu
)\right) \geq \frac{B}{\left( 1-G(y^{\ast })\right) }>0$ donc $w>b+\mu $ et
l'assurance n'est plus compl\`{e}te.

En notant $\Delta _{1}$ et $\Delta _{2}$ les multiplicateurs associ\'{e}s
aux deux contraintes:%
\begin{eqnarray*}
g(y^{\ast })\left( U(b+\mu )-U(w)\right) +\Delta _{1}g(y^{\ast })\left[ -\mu
-y^{\ast }+w\right] -\Delta _{2}g(y^{\ast })\left( U(w)-U(b+\mu )\right) &=&0
\\
\left( 1-G(y^{\ast })\right) U^{^{\prime }}(w)-\Delta _{1}\left( 1-G(y^{\ast
})\right) +\Delta _{2}\left( 1-G(y^{\ast })\right) U^{\prime }(w) &=&0 \\
G(y^{\ast })U^{^{\prime }}(b+\mu )-\Delta _{1}G(y^{\ast })-\Delta _{2}\left(
1-G(y^{\ast })\right) U^{^{\prime }}(b+\mu ) &=&0
\end{eqnarray*}

En combinant la deuxi\`{e}me et la troisi\`{e}me FOC, on a:%
\begin{eqnarray*}
U^{^{\prime }}(w)(1+\Delta _{2}) &=&\Delta _{1} \\
G(y^{\ast })\left( \frac{U^{^{\prime }}(b+\mu )-U^{^{\prime }}(w)}{%
U^{^{\prime }}(b+\mu )}\right) &=&\frac{\Delta _{2}}{1+\Delta _{2}}>0 \\
\text{ et: } &&\Delta _{1}>0
\end{eqnarray*}

et les deux contraintes sont satur\'{e}es.

En particulier, le niveau d'assurance est donn\'{e} par:%
\begin{equation*}
U(w)-U(b+\mu )=\frac{B}{\left( 1-G(y^{\ast })\right) }
\end{equation*}

Plus l'al\'{e}a moral est important $(B\uparrow )$, moins les travailleurs
sont assur\'{e}s ($\mu \downarrow $).

La premi\`{e}re FOC donne:%
\begin{eqnarray*}
\left[ -\mu -y^{\ast }+w\right] &=&\frac{1+\Delta _{2}}{\Delta _{1}}\left(
U(w)-U(b+\mu )\right) \\
&=&\frac{U(w)-U(b+\mu )}{U^{^{\prime }}(w)}
\end{eqnarray*}

d'o\`{u}:%
\begin{eqnarray*}
y^{\ast } &=&w-\mu -\frac{U(w)-U(b+\mu )}{U^{^{\prime }}(w)} \\
&=&b+\left[ w-(b+\mu )-\frac{U(w)-U(b+\mu )}{U^{\prime }(w)}\right]
\end{eqnarray*}

Pour toute utilit\'{e} concave, le terme entre parenth\`{e}se est n\'{e}gatif%
\footnote{%
Plus la curvature est forte, \textit{i.e }plus les travailleurs sont
risque-averses, plus ce terme est n\'{e}gatif} d\`{e}s lors que les
travailleurs ne sont pas parfaitement assur\'{e}s. Donc le seuil optimal est
inf\'{e}rieur \`{a} celui calcul\'{e} dans le cas benchmark. A l'optimum
social, les entreprises gardent des travailleurs alors m\^{e}me que leur
productivit\'{e} est inf\'{e}rieure \`{a} l'\'{e}quivalent salaire
lorsqu'ils ne sont pas employ\'{e}s (moindre efficience): cela traduit un
trade-off entre assurance et efficience: l'al\'{e}a moral recommande une
moindre assurance des travailleurs; du fait de cette moindre assurance, il
est optimal que les firmes licencient moins au prix d'une moindre efficacit%
\'{e}: une baisse du taux de licenciement servant du substitut partiel au
besoin d'assurance (le $y^{\ast }$ obtenu r\'{e}alise de mani\`{e}re optimal
ce trade-off entre assurance et efficience).

\bigskip

10) En d\'{e}duire le design $\{f,\tau \}$ choisi par le gouvernement pour r%
\'{e}aliser l'optimum social en fonction de $w$ et des param\`{e}tres du mod%
\`{e}le. Commenter.

\bigskip

Comme pr\'{e}c\'{e}demment, le gouvernement choisit le couple $\{f,\tau \}$
tel que: $y^{\ast }$ corresponde au choix des entrepreneurs et du
planificateur social et tel que le budget du gouvernement soit \'{e}quilibr%
\'{e}:%
\begin{eqnarray*}
b+\left[ w-(b+\mu )-\frac{U(w)-U(b+\mu )}{U^{\prime }(w)}\right] &=&w+\tau -f
\\
G(y^{\ast })f+\left( 1-G(y^{\ast })\right) \tau &=&G(y^{\ast })\mu
\end{eqnarray*}

donc (en utilisant la contrainte d'incitation des travailleurs):%
\begin{eqnarray*}
f-\mu &=&\tau +\frac{B}{U^{\prime }(w)\left( 1-G(y^{\ast })\right) } \\
\tau &=&-\frac{G(y^{\ast })B}{\left( 1-G(y^{\ast })\right) U^{\prime }(w)}
\end{eqnarray*}

ou encore:%
\begin{eqnarray*}
f-\mu &=&\frac{B}{U^{\prime }(w)} \\
\tau &=&-\frac{G(y^{\ast })B}{\left( 1-G(y^{\ast })\right) U^{\prime }(w)}
\end{eqnarray*}

D\`{e}s que $B>0$, $f>\mu $ et $\tau <0$. Le rapport entre le taux de taxe
aux licenciements et le niveau d'allocation ch\^{o}mage est d\'{e}sormais sup%
\'{e}rieur \`{a} 1: en effet, du fait de l'assurance partielle des ch\^{o}%
meurs, l'optimum exige que les licenciements soient r\'{e}duits, ce qui est r%
\'{e}alis\'{e} en augmentant la taxe aux licenciements. Pour qu'une telle
mesure ne se traduise pas par moins d'emploi qu'\`{a} l'optimum (et que le
budget de l'Etat reste \'{e}quilibr\'{e}), l'Etat subventionne le travail
par une cotisation salariale n\'{e}gative.

En d'autre termes, l'al\'{e}a moral r\'{e}duit l'assurance pour les
travailleurs et pour compenser cette perte d'assurance, l'Etat favorise
l'emploi en taxant les licenciements (et en subventionnant les travailleurs).

\bigskip

Noter que la pr\'{e}sence de $B$ implique que assurance et protection de
l'emploi sont substituts: une hausse de $B$ r\'{e}duit l'assurance des
travailleurs (baisse des allocations $\mu $) du fait de l'al\'{e}a moral et n%
\'{e}cessite donc une meilleure protection de l'emploi (hausse de $f$). Une
mesure visant \`{a} contr\^{o}ler l'effort de recherche des ch\^{o}meurs
(par exemple en organisant un suivi des ch\^{o}meurs afin de les pousser 
\`{a} accepter une offre d'emploi \textquotedblleft
raisonnable\textquotedblright ) peut-\^{e}tre vu comme une baisse de $B$: 
\`{a} l'optimum, une telle mesure doit s'accompagner d'une meilleur
assurance des travailleurs (hausse des allocations) et d'une baisse de la
protection de l'emploi (baisse de $f$) afin d'augmenter l'efficacit\'{e} de
l'\'{e}conomie (meilleure productivit\'{e} des \textquotedblleft
match\textquotedblright\ par une hausse de $y^{\ast }$).

\bigskip

\textbf{C.\ Shallow Pockets}

\bigskip

Jusqu'alors, nous avons suppos\'{e} que les firmes peuvent toujours financer
les taxes de licenciements (pas de contraintes financi\`{e}res,
\textquotedblleft Deep Pockets\textquotedblright ). Nous \'{e}tudions
maintenant le design optimal de la protection de l'emploi lorsque les firmes
sont contraintes financi\`{e}rement.

Nous supposons que la richesse d'un entrepreneur $(W_{E})$ avant de cr\'{e}%
er sa firme est born\'{e}e:

\begin{equation*}
W_{E}\leq I+f^{\ast }
\end{equation*}

\bigskip

11) Quelle est la limite impos\'{e}e \`{a} $f$ dans ce cas? On supposera que
cette contrainte est satur\'{e}e\footnote{%
Dans le cas contraire, le probl\`{e}me est \'{e}quivalent \`{a} la partie A.}%
. En d\'{e}duire que le choix de $\mu $ de l'Etat est contraint par:

\begin{equation*}
G(y^{\ast })\mu \leq f^{\ast }+\left( 1-G(y^{\ast })\right) (y^{\ast }-w)
\end{equation*}

En cas de mauvais \textquotedblleft match\textquotedblright , la firme devra
payer la taxe de licenciement $f$. Elle ne dipose que de $f^{\ast }$ (le co%
\^{u}t fixe \'{e}tant investi). Donc:%
\begin{equation*}
f\leq f^{\ast }
\end{equation*}

Compte-tenu de cette nouvelle contrainte, la contrainte budg\'{e}taire de
l'Etat s'\'{e}crit:%
\begin{eqnarray*}
G(y^{\ast })\mu &\leq &G(y^{\ast })f+\left( 1-G(y^{\ast })\right) \tau \\
&\leq &G(y^{\ast })f^{\ast }+\left( 1-G(y^{\ast })\right) \tau
\end{eqnarray*}

L'Etat cherche \`{a} maximiser le bien-\^{e}tre des travailleurs en tenant
compte du seuil optimal choisi par les entreprises: $y^{\ast }=w+\tau
-f^{\ast }$

Donc le choix de $\mu $ du gouvernement est contraint par:

\begin{eqnarray*}
G(y^{\ast })\mu &\leq &G(y^{\ast })f^{\ast }+\left( 1-G(y^{\ast })\right)
(y^{\ast }+f^{\ast }-w) \\
&\leq &f^{\ast }+\left( 1-G(y^{\ast })\right) (y^{\ast }-w)
\end{eqnarray*}

\textit{Optimum social}

\bigskip

12) Ecrire le programme de maximisation du planificateur social en tenant
compte de cette nouvelle contrainte sur le niveau d'allocations-ch\^{o}mage
et d\'{e}river les conditions du premier ordre (on ne demande pas de r\'{e}%
soudre).\ Montrer que:

a) \`{a} l'optimum, il y a assurance compl\`{e}te

b) le seuil de productivit\'{e} en de\c{c}a duquel il y a licenciement est
plus \'{e}lev\'{e} qu'en A.%
\begin{eqnarray*}
&&\max_{\{w,\mu ,y^{\ast }\}}\left\{ G(y^{\ast })U(b+\mu )+\left(
1-G(y^{\ast })\right) U(w)\right\} \\
s.c &:&\text{ }-G(y^{\ast })\mu +\int_{y^{\ast }}^{\infty }ydG(y)-\left(
1-G(y^{\ast })\right) w\geq I \\
s.c &:&f^{\ast }+\left( 1-G(y^{\ast })\right) (y^{\ast }-w)\geq G(y^{\ast
})\mu
\end{eqnarray*}

En notant $\Lambda _{1}$ et $\Lambda _{2}$ les multiplicateurs associ\'{e}s
aux deux contraintes:%
\begin{eqnarray*}
g(y^{\ast })\left( U(b+\mu )-U(w)\right) +\Lambda _{1}g(y^{\ast })\left[
-\mu -y^{\ast }+w\right] -\Lambda _{2}g(y^{\ast })(y^{\ast }-w+\mu )+\Lambda
_{2}(1-G(y^{\ast })) &=&0 \\
\left( 1-G(y^{\ast })\right) U^{^{\prime }}(w)-\Lambda _{1}\left(
1-G(y^{\ast })\right) -\Lambda _{2}\left( 1-G(y^{\ast })\right) &=&0 \\
G(y^{\ast })U^{^{\prime }}(b+\mu )-\Lambda _{1}G(y^{\ast })-\Lambda
_{2}G(y^{\ast }) &=&0
\end{eqnarray*}

La deuxi\`{e}me et troisi\`{e}me ligne des CPO donnent:%
\begin{eqnarray*}
U^{^{\prime }}(w) &=&\Lambda _{1}+\Lambda _{2}=U^{^{\prime }}(b+\mu ) \\
&\Rightarrow &w=b+\mu
\end{eqnarray*}

et les travailleurs sont parfaitement assur\'{e}s.

La premi\`{e}re CPO donne:%
\begin{equation*}
\left[ -\mu -y^{\ast }+w\right] =-\frac{\Lambda _{2}}{\Lambda _{1}+\Lambda
_{2}}\frac{1-G(y^{\ast })}{g(y^{\ast })}
\end{equation*}

donc:%
\begin{equation*}
y^{\ast }=b+\frac{\Lambda _{2}}{\Lambda _{1}+\Lambda _{2}}\frac{1-G(y^{\ast
})}{g(y^{\ast })}
\end{equation*}

donc $y^{\ast }>b$ si la contrainte de budget est \textquotedblleft
mordante\textquotedblright\ (noter que cette \'{e}quation fixe $y^{\ast }$
de mani\`{e}re implicite).

\bigskip

13) En d\'{e}duire le design $\{f,\tau \}$ choisi par le gouvernement pour r%
\'{e}aliser l'optimum social en fonction de $y^{\ast }$ et des param\`{e}%
tres du mod\`{e}le. Commenter.

\bigskip

L'Etat souhaiterait faire $f=\mu =w-b$ comme dans la partie A.\ Mais du fait
de la contrainte financi\`{e}re qui p\`{e}se sur les entreprises, il doit r%
\'{e}duire $f$ \`{a} $f^{\ast }$. Comme il souhaite tout de m\^{e}me assurer
parfaitement les travailleurs, il est contraint d'utiliser les cotisations
sociales pour financer les allocations ch\^{o}mage et fixe $\tau >0$:%
\begin{eqnarray*}
f^{\ast }-\tau &=&w-y^{\ast }=\mu -\frac{\Lambda _{2}}{\Lambda _{1}+\Lambda
_{2}}\frac{1-G(y^{\ast })}{g(y^{\ast })} \\
G(y^{\ast })\mu &\leq &G(y^{\ast })\left( f^{\ast }-\tau \right) +\tau
\end{eqnarray*}

donc:%
\begin{equation*}
\tau =G(y^{\ast })(\mu -f^{\ast }+\tau )=G(y^{\ast })\frac{\Lambda _{2}}{%
\Lambda _{1}+\Lambda _{2}}\frac{1-G(y^{\ast })}{g(y^{\ast })}>0
\end{equation*}

Pour comprendre l'intuition, supposons que les travailleurs ne sont pas compl%
\`{e}tement assur\'{e}s, en fixant $f=f^{\ast }$ et $\tau =0$ (dans le cas
contraire nous sommes ramen\'{e}s \`{a} la partie A.), \textit{i.e }$w>b+\mu 
$. Pourquoi, le gouvernement est-il incit\'{e} \`{a} lever des cotisations
sociales pour assurer pleinement les travailleurs au d\'{e}triment de
l'emploi? Si le salaire diminue de $\Delta \tau $ et l'Etat pr\'{e}l\`{e}ve $%
\Delta \tau $ comme cotisations sociales, la contrainte de participation des
entreprises n'est pas modifi\'{e} (co\^{u}t du travail identique).

Le budget \'{e}quilibr\'{e} de l'Etat implique que $\mu $ augmente de $%
\Delta \mu =\frac{1-G(y^{\ast })}{G(y^{\ast })}\Delta \tau $ (puisqu'il y a $%
G(y^{\ast })$ ch\^{o}meurs pour $\left( 1-G(y^{\ast })\right) $
travailleurs).

L'utilit\'{e} des ch\^{o}meurs augmente de $U^{^{\prime }}(b+\mu )\Delta \mu
=U^{^{\prime }}(b+\mu )\frac{1-G(y^{\ast })}{G(y^{\ast })}\Delta \tau $.
Celle des travailleurs diminue du fait de la baisse de salaire de $U^{\prime
}(w)\Delta \tau $.

Donc la diff\'{e}rence de bien-\^{e}tre des travailleurs est (en pond\'{e}%
rant par les masses respectives des ch\^{o}meurs et des travailleurs):%
\begin{eqnarray*}
\Delta V_{W} &=&\left[ G(y^{\ast })U^{^{\prime }}(b+\mu )\frac{1-G(y^{\ast })%
}{G(y^{\ast })}-\left( 1-G(y^{\ast })\right) U^{\prime }(w)\right] \Delta
\tau \\
&=&\left( 1-G(y^{\ast })\right) (U^{^{\prime }}(b+\mu )-U^{\prime
}(w))\Delta \tau >0
\end{eqnarray*}

L'Etat est donc bien inciter \`{a} redistribuer du pouvoir d'achat des
travailleurs vers les ch\^{o}meurs tant que ceux-ci ont des revenus inf\'{e}%
rieurs: la raison est simple, l'utilit\'{e} marginale du revenu des ch\^{o}%
meurs est sup\'{e}rieur \`{a} l'utilit\'{e} marginale du revenu des
travailleurs tant que les ch\^{o}meurs ont des revenus inf\'{e}rieurs (\`{a}
la limite, il y a parfaite assurance, les deux effets se compensent). L'Etat
augmente donc les allocations-ch\^{o}mages jusqu\`{a} ce que les
travailleurs soient parfaitement assur\'{e}s.

Noter qu'augmenter $\tau $ augmente le nombre de ch\^{o}meurs mais cela n'a
pas d'impact en terme de bien-\^{e}tre puisque ceux-ci sont parfaitement
assur\'{e}s (et qu'il n'y a pas d'autres b\'{e}n\'{e}fices que ceux du
revenus \`{a} avoir un travail).

\end{document}
