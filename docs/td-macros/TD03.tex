% partie d�clarative
\documentclass[a4paper,10pt]{article}
%\pdfoptionpdfminorversion = 5

\usepackage[english]{babel} %,francais
%\usepackage[latin1]{inputenc}
%\usepackage[T1]{fontenc}\usepackage{amsfonts}
\usepackage{amssymb}
\usepackage{amsmath}
\usepackage{amsthm}
\usepackage{bbm}
\usepackage{stmaryrd}
\usepackage{enumerate}
%usepackage{url}
%usepackage{wasysym}
\usepackage[colorlinks=true,urlcolor=blue,pdfstartview=FitH]{hyperref} 
%usepackage{lscape}
%usepackage{manfnt}
%usepackage{mathbbol}
%usepackage{eurosym}
\usepackage{color}
%\usepackage{bbding}
\usepackage[cyr]{aeguill}     % Police vectorielle TrueType, guillemets fran�ais
\usepackage[pdftex]{graphicx} % Pour l'insertion d'images


%\usepackage{aertt}
\usepackage{harvard}
\usepackage{cje}

%\usepackage[cyr]{aeguill}     % Police vectorielle TrueType, guillemets fran�ais
\usepackage[pdftex]{graphicx} % Pour l'insertion d'images
\DeclareGraphicsExtensions{.jpg,.mps,.pdf,.png,.bmp}%TCIDATA{OutputFilter=LATEX.DLL}
%TCIDATA{Version=4.10.0.2345}
%TCIDATA{Created=Thursday, August 25, 2005 14:35:37}
%TCIDATA{LastRevised=Tuesday, October 18, 2005 15:55:08}
%TCIDATA{<META NAME="GraphicsSave" CONTENT="32">}
%TCIDATA{<META NAME="DocumentShell" CONTENT="Standard LaTeX\Blank - Standard LaTeX Article">}
%TCIDATA{CSTFile=40 LaTeX article.cst}

%\newtheorem{theorem}{Theorem}
%\newtheorem{acknowledgement}[theorem]{Acknowledgement}
%\newtheorem{algorithm}[theorem]{Algorithm}
%\newtheorem{axiom}[theorem]{Axiom}
%\newtheorem{case}[theorem]{Case}
%\newtheorem{claim}[theorem]{Claim}
%\newtheorem{conclusion}[theorem]{Conclusion}
%\newtheorem{condition}[theorem]{Condition}
%\newtheorem{conjecture}[theorem]{Conjecture}
%\newtheorem{corollary}[theorem]{Corollary}
%\newtheorem{criterion}[theorem]{Criterion}
%\newtheorem{definition}[theorem]{Definition}
%\newtheorem{example}[theorem]{Example}
%\newtheorem{exercise}[theorem]{Exercise}
%\newtheorem{lemma}[theorem]{Lemma}
%\newtheorem{notation}[theorem]{Notation}
%\newtheorem{problem}[theorem]{Problem}
%\newtheorem{proposition}[theorem]{Proposition}
%\newtheorem{remark}[theorem]{Remark}
%\newtheorem{solution}[theorem]{Solution}
%\newtheorem{summary}[theorem]{Summary}
%\newenvironment{proof}[1][Proof]{\noindent\textbf{#1.} }{\ \rule{0.5em}{0.5em}}
%\input{tcilatex}

\newcommand{\EnTete}[3]{
		\begin{flushleft}
			Fran\c{c}ois Le Grand \hfill #1
			
			legrand@pse.ens.fr
			
			Cours de Macro�conomie 4 (Prof. Daniel Cohen)
			\url{http://www.pse.ens.fr/junior/legrand/cours.html}
		\end{flushleft}
		\begin{center}	
			\medskip
			\textbf{TD {#2}}
			
			\bigskip
			\textbf{#3} 
		\end{center}

\hrule\vspace{\baselineskip} 
}
%\noindent
\renewcommand{\thesection}{\Alph{section}}
\setlength{\parindent}{0pt}

\makeatletter
\newcounter{question}
\newcounter{subquestion}
\newcommand{\thenumq}{\arabic{question}}
\newcommand{\thenumsq}{\alph{subquestion}}
\newcommand{\ques}{\stepcounter{question}{\hskip.4cm{\thenumq.~\,}}}
\newcommand{\subques}{\stepcounter{subquestion}{\hskip.8cm{\thenumsq.~}}}
\@addtoreset{subquestion}{question}
\makeatother



\newcommand{\textenitalique}[1]{
\bigskip
\textit{#1} \\\nopagebreak
}

\newenvironment{changemargin}[2]{\begin{list}{}{%
\setlength{\topsep}{0pt}%
\setlength{\leftmargin}{0pt}%
\setlength{\rightmargin}{0pt}%
\setlength{\listparindent}{\parindent}%
\setlength{\itemindent}{\parindent}%
\setlength{\parsep}{0pt plus 1pt}%
\addtolength{\leftmargin}{#1}%
\addtolength{\rightmargin}{#2}%
}\item }{\end{list}}

\newcommand{\correction}[1]{
\begin{changemargin}{1cm}{0cm}
   \it #1
\end{changemargin} }


\begin{document}

\EnTete{S�ances des 6--13 Octobre 2006}{3}{Equivalence ricardienne et d\'{e}mographie}

\bibliographystyle{cje}
\bibliography{Biblio_TD}
\nocite{Bu:88}

\paragraph{\large{Points techniques du TD :}} 
\begin{itemize}
	\item \'Equivalence ricardienne,
	\item Lagrangien intertemporel. 
\end{itemize}

On consid\`{e}re une version du mod\`{e}le \`{a} g\'{e}n\'{e}rations en
temps discret, avec incertitude sur la dur\'{e}e de vie. A une p\'{e}riode
donn\'{e}e, la population est constitu\'{e}e de l'ensemble des individus
encore vivants issus des cohortes pass\'{e}es. La d\'{e}mographie est caract%
\'{e}ris\'{e}e par deux param\`{e}tres suppos\'{e}s constants~:

\begin{itemize}
	\item un taux de natalit\'{e} $b$ : \`{a} la p\'{e}riode $t$, une cohorte
de taille $bN_{t}$ vient au monde, $N_{t}$ \'{e}tant la taille de la
population \`{a} la date $t$,
	\item une probabilit\'{e} de d\'{e}c\`{e}s $p$, qui frappe \`{a} chaque p\'{e}%
riode l'ensemble des agents vivants, et ce ind\'{e}pendamment de leur \^{a}%
ge.
\end{itemize}

Par convention, les variables sont d\'{e}finies en d\'{e}but de p\'{e}%
riode.\bigskip 

\section{D\'{e}mographie}

\ques Calculer $n$ le taux de croissance de la population.\medskip

\ques Donner la probabilit\'{e} pour un agent n\'{e} \`{a} la p\'{e}riode $s$
meure avant le d\'{e}but de la p\'{e}riode $t$ $>s$.

\section{Dotations, syst\`{e}me viager et \'{e}volution de la richesse
financi\`{e}re }

On note $a_{s,t}$\ la richesse financi\`{e}re \`{a} la p\'{e}riode $t$ d'un
agent n\'{e} \`{a} la p\'{e}riode $s$. Un agent commence avec une richesse
financi\`{e}re $a_{s,s}=0$. A chaque p\'{e}riode, il re\c{c}oit un salaire $%
w_{t}$ et paie un montant d'imp\^{o}ts $\tau _{t}$. A la p\'{e}riode $t$, un
agent dispose d'un montant $a_{s,t}+w_{t}-\tau _{t}$\ qu'il peut soit
consommer, soit placer.

Le placement de la p\'{e}riode $t$ est doublement r\'{e}mun\'{e}r\'{e} \`{a}
la p\'{e}riode $t+1$ :

\begin{itemize}
	\item au taux d'int\'{e}r\^{e}t r\'{e}el $r$, constant au cours du temps,
	\item via une prime proportionnelle \`{a} son \'{e}pargne.
\end{itemize}

\medskip

On suppose en effet qu'il existe un syst\`{e}me viager parfaitement
concurrentiel (i.e. profit nul) dont le principe est le suivant : \`{a}
chaque p\'{e}riode un agent re\c{c}oit une prime proportionnelle \`{a} son 
\'{e}pargne; en \'{e}change, il r\'{e}troc\`{e}de l'int\'{e}gralit\'{e} de
sa richesse financi\`{e}re au jour de sa mort.\bigskip

\ques Ecrire la loi d'\'{e}volution de la richesse financi\`{e}re d'un individu
n\'{e} \`{a} la p\'{e}riode $s$ pendant la dur\'{e}e de sa vie, \`{a} taux
actuariel\footnote{$\Pi$ est la prime proportionnelle li�e � l'aspect viager} $\Pi $\ donn\'{e}.\medskip

\ques Ecrire l'ensemble des d\'{e}penses et des recettes pour le syst\`{e}me
viager et d\'{e}terminer le taux $\Pi $ v\'{e}rifiant la condition de profit
nul.\medskip

\ques Montrer que le taux d'int\'{e}r\^{e}t (brut) total pour l'\'{e}pargne des
individus est $\frac{1+r}{1-p}$. Par la suite on notera $1+r_{h}\equiv \frac{%
1+r}{1-p}$.

\section{Comportement de consommation optimal}

Un agent appartenant \`{a} la cohorte n\'{e}e \`{a} la p\'{e}riode $s$
cherche \`{a} maximiser 
\begin{equation*}
U_{s}=\sum_{t=s}^{\infty }(1-p)^{t-s}\beta ^{t-s}\log (c_{s,t}{)}
\end{equation*}

L'\'{e}volution de la richesse est contrainte par la condition suivante : 
\begin{equation*}
\lim_{t\rightarrow \infty }\quad \frac{a_{s,t}}{(1+r_{h})^{t}}=0
\end{equation*}

\ques Ecrire le programme de maximisation d'un agent n\'{e} \`{a} la p\'{e}%
riode $s$ et le Lagrangien associ\'{e}.\medskip

\ques D\'{e}river les conditions du premier ordre\ et montrer que : 
\begin{equation*}
\frac{c_{s,t+1}}{c_{s,t}}=\beta (1+r)
\end{equation*}

\ques En utilisant la condition de transversalit\'{e}, \'{e}crire la
contrainte budg\'{e}taire intertemporelle. On notera : 
\begin{equation*}
h_{t}=\sum_{i=0}^{\infty }\frac{(w_{t+i}-\tau _{t+i})}{(1+r_{h})^{i}}
\end{equation*}

\ques D\'{e}terminer $c_{s,t}$\ pour $t\geq s$ en fonction de $a_{s,t}$ et de $%
h_{t}$.\medskip

\ques Ecrire $C_{t}$\ la consommation agr\'{e}g\'{e}e sur l'ensemble de la
population.

\section{Finances publiques}

L'Etat r\'{e}alise des d\'{e}penses $G_{t}$ (qui n'apparaissent pas dans la
fonction d'utilit\'{e} des agents), qu'il peut financer soit
par l'imp\^{o}t (on note $T_{t}$ les pr\'{e}l\`{e}vements fiscaux agr\'{e}g%
\'{e}s), soit par la dette, sur laquelle il paie un taux d'int\'{e}r\^{e}t $%
r $.\bigskip

\ques Ecrire la loi d'\'{e}volution de la dette $D_{t}$.\medskip

\ques L'\'{e}volution de la dette est contrainte par la condition suivante : 
\begin{equation*}
\lim_{t\rightarrow \infty }\quad \frac{D_{t}}{(1+r)^{t}}=0
\end{equation*}
Montrer qu'on doit avoir : 
\begin{equation*}
\sum_{i=0}^{\infty }\frac{T_{t+i}}{(1+r)^{i}}=D_{t}+\sum_{i=0}^{\infty }%
\frac{G_{t+i}}{(1+r)^{i}}
\end{equation*}%


\section{Equilibre et condition pour la neutralit\'{e} de la dette}


\ques Expliquer pourquoi on doit avoir \`{a} chaque p\'{e}riode $A_{t}=D_{t}$%
.\medskip

\ques Substituer dans $C_{t}$ l'expression de $D_{t}$\ en fonction des exc\'{e}%
dents budg\'{e}taires futurs. \medskip

\ques Montrer que la consommation est ind\'{e}pendante du chemin de taxe choisi
si et seulement si : 
\begin{equation*}
b=0
\end{equation*}

\ques Dans cette \'{e}conomie, l'argument selon lequel il n'y a pas \'{e}%
quivalence ricardienne parce qu'au moment de rembourser la dette on sera
peut-\^{e}tre mort est-il valable?

\end{document}
